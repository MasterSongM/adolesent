
\color{olive}



5.3. 在线社交的动机、涉及的认知能力?

5.4.社交成瘾

   1. 社交成瘾的定义 及表现

    2. 社交成瘾如何影响认知功能?其机理是怎样的? (双系统,抑制控制能力受损,。。。)

5.5 结论

干预 (放到第七章)



\section{ 在线社交兴起的背景及对人们工作生活的影响  }


 (可以参考这本书 Understanding Social Media Logic, \cite{DijckPoell2013})

=====

(Ref: Older Adolescents’ Motivations for Social Network Site Use: The Influence of Gender, Group Identity, and Collective Self-Esteem
Valerie Barker, Ph.D thesis )

%Social identity gratifications (SIG): Opportunities to identify with ingroup members who look and act similarly to each other as well as to compare themselves to outgroup members. Apart from passing time, entertainment, and information seeking, interpersonal/interactive goals are consistently identified. 

\begin{enumerate}
\item 社会认同感(Social identity gratifications, SIG):与内部成员相识的机会,这些成员看起来和行为相似,并将自己与外群成员进行比较。除了消磨时间,娱乐和寻求信息之外,还有人际关系/互动目标。

在做判断以及做事情的时候,我们常常会根据别人的意见而做出改变,尤其是当我们对一件事不确定或者没有把握的情况下,他人对我们的影响就会更大。类似于从众心理,它们之间的不同点在于,从众是对大多数的服从和跟随,而社会认同不仅会考虑别人的行为和看法,还会服从于常规的社会行为规范。


\item 
%Social compensation : Individuals who experience low or negative collective self-esteem—perceiving their social group to be an undesirable/unpopular one—may wish to distance themselves from it, particularly if they believe that others evaluate the group negatively. Perhaps they feel the need to move on, and in such situations these persons may try to develop more rewarding group relationships online— a form of social compensation.
社会补偿 (Social compensation):体验低或负集体自尊的个体认为他们的在社会群体中是一个不受欢迎/不受欢迎的人,特别是如果他们认为其他人进行负面评价,可能希望与社会群体保持距离。然而在他们认为需要让生活前进的情况下,这些人可能会尝试在线发展这种更有价值的团体关系 - 这是一种社会补偿现象。

\item %  Gender differences in SNS use : Boys focus on features and entertainment; girls seem more interested the relational aspects of social media.  Girls are more likely than boys to talk with friends on the Internet about romantic relationships, secret things, and deep feelings.While girls use SNS to maintain contact with their friends, boys were more likely to use their sites to make new friends
SNS使用中的性别差异:男孩注重功能和娱乐;女孩似乎对社交媒体的关系方面更感兴趣。女孩比男孩更有可能在网上与朋友谈论浪漫的关系,秘密的事情和深刻的感情。虽然女孩使用SNS与朋友保持联系,男孩更有可能利用他们的网站结交新朋友。

\end{enumerate}

%most of the participants who reported high collective self-esteem also reported SNS use to communicate with peer group members.  Positive collective self-esteem was also strongly related to entertainment and passing time.Those who reported negative collective self-esteem reported more instrumental interest in SNS use for social compensation, learning, and SIG. 
高集体自尊的参与者将SNS用于与同伴小组成员交流。积极的集体自尊也与娱乐和消磨时间密切相关。那些消极的集体自尊的人将SNS用于社会补偿,学习和SIG方面。
 
%Among those with high positive collective self-esteem, females were more likely to use SNS to communicate with peers, pass time,and entertain while males were more likely to use SNS to seek social compensation, SIG, and learning. Among those with high negative collective self-esteem, females compared to males reported using SNS to pass time and entertain; the means were lower than those for who reported high positive collective self-esteem. 
在具有高度积极的集体自尊的人中,女性更有可能使用SNS与同龄人交流,消磨时间和娱乐,而男性更有可能使用SNS寻求社会补偿,SIG和学习。在那些具有较高的负面集体自尊的人中,相比男性,女性使用SNS消磨时间和娱乐;手段低于高度积极的集体自尊的人。

%Those who felt less secure in face-to-face interaction were more likely to turn to the Internet for interactional purposes, those who reported a disconnect from their peer group were more likely to seek social compensation and SIG via SNS. Older adolescents who feel isolated and exhibit negative collective self-esteem seem to turn to their SNS for companionship. They may desire to identify with others by using their SNS because they do not have positive relationships with ingroup members in their everyday lives. 
那些在面对面交互中感觉不太舒服自在的人更有可能为互动目的而转向互联网,那些与同龄人群脱节的人更有可能通过SNS寻求社会补偿和SIG。感到孤立并表现出消极的集体自尊的年长青少年似乎转向他们的SNS进行陪伴。他们可能希望通过使用他们的SNS与他人认同,因为他们在日常生活中与内心成员没有积极的关系。

\section*{Research Publication No. 2007-16 December 2007 
      Why Youth Heart Social Network Sites:  The Role of Networked Publics in Teenage Social Life 
Danah Boyd}

网众传播networked public(传播学): 网众传播是一种现实层面的新范式,由“网众”发起和参与,由“社会性媒体”中介的传播模式、现象与行为,体现在信息流动、人际关系形成发展、社会网络化和权力互动模式等各方面的变化。

%Inunmediated environments: The boundaries and audiences of a given public are structurally defined. Access to visual and auditory information is limited by physics; walls and other obstacles further restrain visibility.  Mediating technologies: a mediated public (and especially a networked public) could consist of all people across all space and all time.Eg: television, radio, and newsprint change everything networked public
不受媒体干扰的环境:
特定公众的界限和受众在结构上被定义。获取视觉和听觉信息受物理限制;墙壁和其他障碍进一步限制了能见度。
媒体调解技术:
一个中介的公众(尤其是网络公众)可以由所有空间和所有时间的所有人组成。例如:电视,广播和新闻纸改变一切
网络化公众

%four properties of networked public: 1 Persistence: Unlike the ephemeral quality of speech in unmediated publics, networked communications are recorded for posterity. This enables asynchronous communication, but it also extends the period of existence of any speech act.  2 Searchability: Because expressions are recorded and identity is established through text, search and discovery tools help people find like minds. While people cannot currently acquire the geographical coordinates of any person in unmediated spaces, finding one’s digital body online is just a matter of keystrokes.  3 Replicability: Hearsay can be deflected as misinterpretation, but networked public expressions can be copied from one place to another verbatim such that there is no way to distinguish the “original” from the “copy.” 4 Invisible audiences: While we can visually detect most people who can overhear our speech in unmediated spaces, it is virtually impossible to ascertain all those who might run across our expressions in networked publics .This is further complicated by the other three properties, since our expression may be heard at a different time and place from when and where we originally spoke. 
网络公众的四个属性:
1持久性:与无媒体干扰公众的短暂性不同,为后代记录网络通信。这使得异步通信成为可能,它也延长了任何言语行为的存在期。
2可搜索性:因为表达式被记录并且身份通过文本建立,所以搜索和发现工具可以帮助人们找到像思想一样的东西。虽然人们目前无法在无中介空间中获取任何人的地理坐标,但在线查找一个人的数字机构只是一个按键问题。
3可复制性:传闻可以被解释为误解,但网络公共表达可以从一个地方逐字复制到另一个地方,这样就无法区分“原始”和“副本”。
4隐形观众:虽然我们可以在视觉上检测到大多数能够在无媒体中介空间中听到我们演讲的人,但几乎不可能确定所有那些可能在网络公众中遇到我们表达的人。由于我们的其他三个属性,这使其进一步复杂化。可以在我们最初说话的时间和地点的不同时间和地点听到表达。

%Initiation: Profile Creation: By lookingatothers' profiles,teens get a sense of what types of presentations are socially appropriate; others’ profiles provide critical cues about what to present on their own profile. While profiles are constructed through a series of generic forms, there is plenty of room for them to manipulate the profiles to express themselves. At a basic level, the choice of photos and the personalized answers to generic questions allow individuals to signal meaningful cues about themselves.  开始:建立资料
初始化:通过观察其他人的资料,青少年可以了解哪种类型的呈现适合社交;其他人的专业知识提供了关于在他们自己的专业中呈现什么的关键线索。虽然资料是通过一系列通用形式构建的,但是他们有足够的空间来修改资料来表达自己。在基本层面上,照片的选择和通用问题的个性化答案可以让个体来彰显自己的特征和意义

%Identity Performance  : What we put forward is our best effort at what we want to say about who we are. Yet while we intend to convey one impression, our performance is not always interpreted as we might expect. Through learning to make sense of others’ responses to our behavior, we can assess how well we have conveyed what we intended. We can then alter our performance accordingly.  This process of performance, interpretation, and adjustment is what Erving Goffman calls impression management(印象操纵)
身份表现:
我们尽可能的表达出我们是谁,我们的个人特征。然而,虽然我们打算传达一种印象特征,但我们的表现并不总是按照我们的预期来传达。通过学习理解他人对我们行为的反应,我们可以评估我们如何表达我们的意图。然后我们可以相应地改变我们的表现,这种表现,解释和调整过程是Erving Goffman所说的印象管理(印象操纵)

%Impression management is a conscious or subconscious process in which people attempt to influence the perceptions of other people about a person, object or event. They do so by regulating and controlling information in social interaction.
印象管理是一种有意识或潜意识的过程,在这种过程中,人们试图影响他人对某个人,物体或事件的看法。他们通过调节和控制社交互动中的信息来做到这一点。







\section{在线社交的定义、特征 %(郝洁)
}


%社交媒体的定义
社交媒体是交互式计算机媒介技术,通过虚拟社区和网络促进信息、思想、职业兴趣和其他形式表达的创建和共享。它是基于用户关系的内容生产与交换平台,人们彼此之间用它来分享意见、见解、经验和观点。现阶段的社交媒体主要包括社交网站、微博、微信、博客、论坛、播客(包括抖音等短视频)等等。
社会媒体和一般的社会大众媒体最显著的不同是,让用户享有更多的选择权利和编辑能力,自行集结成某种阅听社群,并能够以多种不同的形式来呈现,包括文本、图像、音乐和视频。传统媒体,也称为广播媒体,通常由已建立的制作源(例如电影制片厂,电视网络或编辑人员)在外部创建,并且被提供给个人或更广泛的观众以用于被动观看或阅读。相比之下,包括社交和交互式媒体在内的较新的数字媒体是一种媒体形式,用户可以在其中消费并积极创建内容。示例包括应用程序(应用程序),多人视频游戏,YouTube视频或视频博客(视频博客)。对于今天的儿童和年轻人来说,被动观看和互动媒体的这种不断发展的整合是无缝和自然的;传统/广播和互动/社交媒体之间的区别和界限变得模糊或难以察觉。
同样作为信息交流的工具,与传统媒体不同的是,社交媒体可以是多向的,它基于了社交网络,人们的消息在网络中传递。比如A认识B,B认识C,那么C就存在于A的社交网络之中,而在B的社交网络之中,A和C都与其直接相连。有人结合社交和媒体各自的特点,将社交媒体这样定义:社交媒体是基于网络的通信工具,通过共享和消费信息,人们可以相互交流。
%社交媒体特征
一,用户帐户(代表个体):如果某个网站允许访问者创建自己可以登录的帐户,那么这是一个好的迹象,那就是社交互动。 如果不通过用户帐户进行操作,您无法真正在线共享信息或与他人互动。
二,个人资料页面(展示自我):由于社交媒体都是关于沟通的,因此通常需要个人资料页面来代表个人。 它通常包括有关个人用户的信息,如个人资料照片,生物,网站,最近帖子的提要,推荐,近期活动等。
三,朋友,粉丝,群组,主题标签等(社交网络):个人使用他们的帐户与其他用户建立联系。 他们还可以使用它们订阅某些形式的信息。
四,新闻提要(流向自己的信息):当用户在社交媒体上与其他用户联系时,他们基本上会说,“我想从这些人那里获取信息。” 该信息通过他们的新闻源实时更新。
五,个性化(选择性、个性):社交媒体网站通常可以让用户灵活地配置他们的用户设置,自定义他们的个人资料以查看特定的方式,组织他们的朋友或关注者 ,管理他们在新闻提要中看到的信息,甚至提供他们做什么或做什么的反馈我不想看。
六,通知:任何通知用户有关特定信息的网站或应用程序肯定在玩社交媒体游戏。 用户可以完全控制这些通知,并可以选择接收他们想要的通知类型。
七,信息更新,保存或发布:如果网站或应用程序允许您发布任何内容,无论是否有用户帐户,那么它是社交的! 它可以是简单的基于文本的消息,照片上传, YouTube视频 ,文章链接或其他任何内容。
八,就像按钮和评论部分一样(态度、想法):我们在社交媒体上互动的两种最常见的方式是通过代表“喜欢”的按钮加上评论部分 ,我们可以分享我们的想法。
九,评论,评级或投票系统(态度、想法):除了喜欢和评论,许多社交媒体网站和应用程序依靠社区的集体努力来审查,评价和投票他们知道或使用过的信息。 想想您最喜欢的购物网站或使用此社交媒体功能的电影评论网站 。

%以下为11.10分享内容,可以写在社交成瘾
双系统模型是类比已有的类似的成瘾的研究。(“调整了Collins和Lapp [23]的酒精模型” )
提出四个假说
            H1:SNS的问题使用与学生的学习成绩呈负相关。
            H2:使用SNS的认知 - 情绪专注与SNS的有问题使用正相关。
            H3:使用SNS时的认知 - 行为控制与SNS的有问题使用负相关。
            H4:使用SNS的认知 - 行为控制降低了使用SNS的认知 - 情绪关注与SNS的有问题使用之间的关系的强度。
“这指出了这些行为在很大程度上是非理性的基础,正如本研究所证明的那样,它们可以通过理论上的解释来更好地解释,例如双系统理论,它可以合理地揭示其潜在的病因。借助双系统理论,这项研究表明,与许多其他有问题的行为一样,有问题的SNS使用是由强烈的认知 - 情绪专注于行为加上弱特定控制能力所引发的。”
“这项研究有助于IS使用和双系统理论的文献。在并非所有使用实例都经过充分计划和理性的情况下,与系统行为模型(也就是之前有人提过另一种解释模型planned behavior models)相比,双系统理论可以指出能够更好地解释这种自发和有问题行为的相关因素。 这是IS使用文献的重要延伸,主要依赖于合理性假设[10]。”\cite{he2017brain}


该研究依赖于关于控制成瘾和过度行为的双系统组分的神经可塑性的知识,并且表明特定感兴趣区域的灰质体积(即脑形态)的改变与技术相关的成瘾相关。使用基于体素的形态测量法(VBM)应用于具有不同程度的SNS成瘾的20个社交网站(SNS)用户的结构磁共振成像(MRI)扫描,我们显示SNS成瘾与可能更有效的冲动脑系统相关联,表现出来通过双侧减少杏仁核中的灰质体积(但不是伏隔核的结构差异)。在这方面,SNS成瘾在大脑解剖学改变方面类似于其他(物质,赌博等)成瘾。我们还表明,与前扣带/中扣带皮层受损并且不能支持所需抑制的其他成瘾形成对比,后者通过减少的灰质体积表现出来,该区域在我们的样本中被认为是健康的并且其灰色物质量与一个人的SNS成瘾水平正相关。这些发现描绘了SNS成瘾的解剖学形态学模型,并指出了脑形态学在成瘾与物质和赌博成瘾之间的相似性和差异性。\cite{he2017brain}

与过度和强迫性社交媒体使用相关的成瘾样症状在一般人群中是常见的。由于它们可能导致各种不良反应,因此越来越需要了解潜在社交媒体成瘾所涉及的大脑系统和过程。我们专注于岛状皮层(即岛叶)后部细分的形态,因为它已被证明有助于维持物质成瘾和有问题的行为。假设社交媒体成瘾与更成熟的成瘾共享神经相似性,并与神经经济学领域的证据相一致,我们进一步研究了这种关联的一个可能原因 - 即岛屿形态影响一个' 推迟折扣,这种延迟折扣导致对即时社交媒体奖励和随之而来的成瘾症状的夸大偏好。基于基于体素的形态测量技术应用于32位社交媒体用户的MRI扫描,我们发现双侧后脑岛的灰质体积与社交媒体成瘾症状呈负相关。我们进一步表明,这种关联是通过延迟折扣来调节的。这提供了初步证据,证明岛屿形态可能与潜在的社交媒体成瘾有关,部分原因是它通过延迟折扣所带来的对不良远见和冲动的贡献。\cite{turel2018delay}

%以下10.20分享,可以写在社交与认知控制部分
“中介效应分析的结果表明,社交网站使用对自我概念清晰性不仅有显著的直接预测效应,还能通过社会比较倾向的间接作用对自我概念清晰性产生影响。首先,社交网站使用对自我概念清晰性有显著的负向预测作用,这一结果与之前一些研究者的观点并不一致。如前所述,社交网站为青少年提供了一个进行自我探索的理想空间,在网络中个体可以方便表达和探索自我的不同方面,给个体提供了塑造、扮演自我的机会,并通过与他人的互动反馈来进行自我反思和自我建构 (Subrahmanyam \& \v{S}mahel, 2011; Walther et al., 2011),但本研究的结果却发现,社交网站的使用不利于个体形成稳定清晰的自我认知。这也进一步说明社交网站的使用会对个体自我的发展产生消极影响,使个体面临自我不同方面无法统合的风险 (Davis, 2013; Valkenburg \&Peter, 2011)。”
“此外,社交网站使用还能通过社会比较倾向的间接作用对自我概念清晰性产生影响,这一机制在在社交网站对个体自尊和身体自我意象的影响中也得到了证实 (Tiggemann \& Slater, 2013; Vogel et al.,2014)。个体不仅会在社交网站中呈现大量自我相关的信息,也常常会通过社交网站来获取有关他人的信息 (Vogel et al., 2014),因此,个体在社交网站中也会不可避免地成为其他用户呈现内容的受众,这就会成为诱发个体进行社会比较的情景因素;再加上社交网站中的好友绝大多数是个体线下的朋友、同学 ( 中国互联网信息中心 , 2014),这种相似群体的信息更能诱发个体的社会比较倾向 ( 郭淑斌 , 黄希庭 , 2010)。而社会比较会使个体过多地关注外界信息并依据外界信息来进行自我评价和自我认知,从而动摇对自我的评价和看法,进一步对个体的自我概念清晰性产生消极影响 (Butzer \& Kuiper, 2006;Vartanian \& Dey, 2013)。”
“社交网站使用对自我概念清晰性的负向预测作用以及社会比较在其中的中介作用,提示我们要客观看待社交网站的影响,社交网站旨在维系和扩展人际关系网络,虽然有助于个体的心理社会适应水平的提高,却会对个体自我的发展产生消极影响,这看似矛盾的结果反映了社交网站对个体不同方面的影响。这也提示我们要进一步深入研究社交网站使用对青少年发展的影响,并重视社交网站使用和个体适应关系中的其它影响变量。同时,研究者也指出不同的社交网站使用行为对个体的影响可能不同 ( 姚琦等 , 2014; Bevan, Gomez, \& Sparks, 2014),但本研究仅仅探讨了一般性社交网站使用对自我概念清晰性的影响,未来的研究也很有必要去考察特定的社交网站使用行为对个体心理社会适应和自我发展的影响。最后,这一结果启示我们要引导青少年合理使用社交网站,客观合理看待社交网站中的好友信息,以期使社交网站对青少年的发展发挥积极作用。”\cite{牛更枫2016青少年社交网站使用对自我概念清晰性的影响}

“儿童对于互联网的使用主要包括浏览网页、游戏以及在线沟通。浏览网页涉及的认知过程包括文本理解和图像识别,因此儿童浏览互联网能丰富知识库,获得概念的发展。信息搜索的过程也会促进元认知功能中的计划、策略搜寻和信息评估的发展。电子游戏需要儿童同时监控多种视觉刺激、纵观图形变化、觉察各种视觉空间之间的关系等,这些同时性加工的操作需要更高的视觉能力和元认知技能,因此电子游戏能够促进儿童空间感知和视觉注意等方面的发展( Johnson,2010)。研究者利用游戏能够提供快速及时的反馈,紧密追踪儿童能力反应水平的特点,设计训练认知能力的游戏( Diamond \& Lee,2011),此类训练已经证明能够提高不同年龄群体和某些特殊儿童的工作记忆能力( Nutley et al.,2011)。”
“近期的研究表明,媒体的使用方式也会对个体的认知产生影响。媒体多任务操作( media multi-tasking)指的是在媒体使用的一个时间段内,同时进行多种媒体或者非媒体活动。研究表明经常进行多任务同时操作的个体过滤无关信息的能力更差,任务转移的能力更低( Ophir,Nass,& Wagner,2009);但多任务操作者之所以更易被分心可能是因为他们的注意范围更广,他们分析了那些无关的但可能有潜在意义的干扰项目 ( Lin,2009)。有研究表明有分心物存在时或者进行多任务操作时会改变大脑的学习过程,多任务操作的学习过程激活了一般学习过程中没有活动的纹状体,而正常学习激活的海马则不再活跃( Poldrack & Foerde,2008)。”
“早期研究表明,互联网会减少面对面交往,削弱个体和社会的联接( Kraut et al.,1998)。后期多项研究表明电子产品可以促进社会交往,扩大个体的社交网络,增进关系的亲密程度( Lee,2009; Valk-enburg & Peter,2009,2011)。这种促进作用有两种表现方式: 富者更富和社会补偿 ( Kraut et al.,2002)。富者更富假说认为,具有更好社会网络和社交技能的个体可以从互联网交往中获得更多的益处( Desjarlais & Willoughby,2010; Gross,Juvonen,&Gable,2002)。社会补偿假说认为,线上交流能弥补那些社会焦虑或者社会疏离个体的社交网络( Young & Lo,2012),内向的个体、害羞的个体能够更好克服性格缺点,在虚拟世界中更从容地建立社会关系 ( Bessière, Pressman, Kiesler,& Kraut,2010)。但是当那些在现实交往中处于劣势的青少年将网络做为逃避现实的手段则会带来更严重的后果,他们可能沉溺于虚拟世界,也即穷者更穷 ( Self-hout,Branje,Delsing,ter Bogt,& Meeus,2009)。”\cite{杨晓辉2014电子媒体的使用与儿童发展}

%9.28分享,可以分在认知部分
%Despite a huge spike in smartphone overuse, the cognitive and emotional consequences of smartphone overuse have rarely been examined empirically. In two studies, we investigated whether separation from a smartphone influences state anxiety and impairs higher-order cognitive processes, such as executive functions. We found that smartphone separation causes heightened anxiety, which in turn mediates the adverse effect of smartphone separation on all core aspects of executive functions, including shifting (Experiment 1) and inhibitory control and working-memory capacity (Experiment 2). Interestingly, impaired mental shifting was evident regardless of the extent of smartphone addiction, whereas smartphone addiction significantly moderated the negative effect of smartphone separation on inhibitory control, as assessed by the Stroop task. The study sheds light on cognitive mechanisms that may underlie some of these negative consequences of smartphone overuse.
尽管智能手机的过度使用数量激增,但智能手机过度使用的认知和情感后果很少得到实证检验。在两项研究中,我们调查了与智能手机的分离是否会影响状态焦虑,损害高级认知过程,比如执行功能。我们发现,智能手机的分离会导致焦虑加剧,这反过来又会影响智能手机分离对执行功能的所有核心方面的负面影响,包括转移(实验1)和抑制控制以及工作记忆能力(实验2)。有趣的是,无论智能手机成瘾程度如何,心智转移(认知转移)障碍都很明显,而根据Stroop任务的评估,智能手机成瘾显著降低了智能手机分离对抑制控制的负面影响(在抑制控制方面,智能手机成瘾显著缓解了智能手机分离的负面影响?)。这项研究揭示了一些认知机制,这些认知机制可能是过度使用智能手机的负面后果。\cite{hartanto2016smartphone}

(Ref:《Are Smart-phones Really Destroying the Adolescent Brain?》)
%本文原文引用案例较多,原文长

%The subjects were more likely to “like” pictures if they believed the images were already popular. They also showed more activation in regions involved in social cognition and visual attention, as though they were thinking more about the highly liked pictures and scrutinizing them. When the subjects’ own photos received a lot of “likes,” they showed a response in the ventral striatum, a brain region involved in reward. “That might explain why teens are particularly avid users of social media and why they find it so motivating,” Sherman says.
社交媒体的点赞对青少年是一种激励。

%Long-term research that could show causality would be hard to execute. “You can’t randomly assign kids to have a phone or not,” Steinberg says. Studying teenagers requires obtaining their parents’ permission—an extra logistical challenge. That means that expert predictions are often extrapolations of research on college students. “Sometimes we have good reason to think that the findings from research on young adults may generalize to younger teens, but we have no way of knowing for sure.” Complicating matters is the finding that some brain structures, such as the prefrontal cortex, don’t fully develop until the mid-20s.//Another research design challenge hinges on what exactly a “smartphone” is. It’s a telephone, a camera, a game console and an encyclopedia. Even zeroing in on specific apps teens favor, such as Snapchat and YouTube, is insufficiently broad. “When you’re asking how kids are being affected by social media,” Stein-berg says, “it’s like asking about the effect of TV without distinguishing between Jersey Shore and Masterpiece Theatre.”
青少年-社交媒体的研究设计极具挑战,其中涉及因素众多:比如家长是否准许,比如长期研究中脑结构的影响,比如究竟什么是社会媒体,而你的研究中的社会媒体是否太过宽泛。

%Larry D. Rosen, a professor emeritus of psychology at California State University, Dominguez  Hills, and co-author of the 2016 The Distracted Mind: Ancient Brains in a High-Tech World, suspects that while it is possible that kids who are already de pressed or anxious use smartphones differently, the influence probably goes both ways. Rosen thinks that social comparison (where social media browsers feel awful about their lives after getting bombarded with rosy versions of everyone else’s) and emotional contagion (where negative online outbursts affect browsers’ states of mind) are possible culprits. Whether or not a teen experiences a self-esteem dip or secondhand moodiness comes down to who they’re associating with on -line and what exactly they’re looking at.

关注社交平台使用时的社交比较(别人丰富多彩的朋友圈)和情绪感染(很多负面性表达)可能与孩子焦虑、抑郁有关.

%It is that precise aspect of how social media is used that researchers are now testing. Oscar Ybarra of the University of Michigan and his colleagues found that subjective well-be ing was negatively affected by passive use of social media sites because comparisons sparked envy. But active use—posting content and interacting with others rather than just “lurking”—predicted higher levels of subjective well-being, seemingly because active use creates social capital and makes users feel more connected to other people.

社交比较会引发嫉妒,但社交媒体的积极使用(发布内容并与他人互动)会预测更高水平的主观幸福感。

% Temple psychologists Harry Wilmer and Jason Chein found a correlation between heavier smartphone use and less of an ability to delay gratification, for example, taking a smaller sum of money in the moment rather than waiting for a larger amount.
 
 较多的智能手机使用与较少的延迟满足能力之间存在相关性,例如,在此刻花费较少的钱而不是等待更大的金额。

%4.For Rosen, a big concern is not just how teens are using their phones but rather the “technological anxiety” and nomophobia (the feeling someone gets in the absence of their phone), that distract them from other tasks. Research has shown that multitasking leads to worse performance on any of the tasks in play. Using an app, Rosen monitored how many times his students unlocked their phone each day. “It was 50 times, on average,” he says, “and they stayed on the phone for about five and a quarter minutes each time.” Most of the near-constant checking in had to do with communication because their top apps were Facebook, Instagram, Snapchat and YouTube. “We know that half of the time people check in, it’s because they get an alert or notification.” Adults seem to be affected, too: a British study showed that just the presence of a phone on a table between two people chatting about a meaningful topic had a negative effect on closeness and conversation quality. The call of the phone is cognitively loud, even when it’s turned off.

更令人担心的是技术焦虑(手机焦虑),关注手机会分散青少年对其他任务的注意力。研究表明,多任务处理会导致处理每个任务的性能下降。英国的一项研究表明,两个人聊一个有意义的话题时,仅是桌子上有一个电话,对亲密度和会话质量有负面影响。更更令人担心的是暂时焦虑造成的长期脑损伤。可能机制:看手机->皮质醇增多->引起焦虑->为了平息焦虑看手机。

%Even the assumption that face-to-face interactions are more satisfying and profound is not always true. Sherman asked her subjects whether there are certain topics they feel more comfortable talking about via digital communications such as texting. They said that if they wanted to say something really emotional and felt like they might cry, they preferred texting. Particularly because they are often interacting online with real-life friends, a different and maybe even deeper mode of bonding can take place as teens trade disclosures that are difficult to say out loud.
有时有些话题短信通信更自在(短信的非即时性,且避免面对面一些额外影响)。

%Sakari Lemola, an assistant professor of psychology at the University of Warwick in England, recently found that teenagers with smartphones fall asleep later at night. “This is probably be cause they’re engaging with social media, communicating with friends and watching YouTube,” Lemola says. “We also found that electronic media use around bedtime was related to de creased sleep duration and increased symptoms of insomnia. Short sleep and poor sleep quality were in turn related to de -pressive symptoms.”
%There are several possible connections, Lemola says. Modern flat screens emit a larger amount of blue light, which suppresses melatonin, a hormone produced by the pineal gland at night or in the dark that regulates our internal clocks. Getting messages or comments from friends on social media is arousing for teens and makes it more difficult for them to fall asleep. And it’s hard to shut off the phone when endless entertainment beckons.
与更少的睡眠有关:蓝光->抑制褪黑激素->内部失调->继续看手机

%Social media use appeared to improve their ability to understand—and to share the feelings of—their peers during that time frame.//It has been said that social media brings out the worst in teens—and even impairs their social functioning. Dutch researchers Helen G. M. Vossen and Patti M. Valkenburg tested this idea. They surveyed 942 people ages 10 to 14 and gave them a test called the Adolescent Measure of Empathy and Sympathy (AMES), then did the same evaluation a year later. They found that social media use increased over the year, along with the teens’ ability to understand and share their peers’ feelings.
与更好的共情能力有关:

社交媒体定义:%引自维基百科
%Social media are interactive computer-mediated technologies that facilitate the creation and sharing of information, ideas, career interests and other forms of expression via virtual communities and networks.[1] The variety of stand-alone and built-in social media services currently available introduces challenges of definition; however, there are some common features:[2]
   % 1)Social media are interactive Web 2.0 Internet-based applications.[2][3]
    %2)User-generated content, such as text posts or comments, digital photos or videos, and data generated through all online interactions, is the lifeblood of social media.[2][3]
    %3)Users create service-specific profiles for the website or app that are designed and maintained by the social media organization.[2][4]
    %4)Social media facilitate the development of online social networks by connecting a user's profile with those of other individuals or groups.[2][4]
社交媒体是互动的计算机媒介技术,通过虚拟社区和网络促进信息,思想,职业兴趣和其他形式表达的创造和共享。 目前可用的各种独立和内置社交媒体服务引入了定义的挑战; 但是,有一些共同的特点:
     1.社交媒体是基于互联网的Web 2.0基于互联网的应用程序。
     2.用户生成的内容,如文本帖子或评论,数码照片或视频,以及通过所有在线互动产生的数据,是社交媒体的生命线。
     3.用户为社交媒体组织设计和维护的网站或应用程序创建特定于服务的配置文件。
     4.社交媒体通过将用户的个人资料与其他个人或群体的个人资料相关联来促进在线社交网络的发展。
[1]Kietzmann, Jan H.; Kristopher Hermkens (2011). "Social media? Get serious! Understanding the functional building blocks of social media". Business Horizons. 54 (3): 241–251. doi:10.1016/j.bushor.2011.01.005.
[2]Obar, Jonathan A.; Wildman, Steve (2015). "Social media definition and the governance challenge: An introduction to the special issue". Telecommunications policy. 39 (9): 745–750. doi:10.1016/j.telpol.2015.07.014. SSRN 2647377 
[3]Kaplan Andreas M., Haenlein Michael (2010). "Users of the world, unite! The challenges and opportunities of social media" (PDF). Business Horizons. 53 (1): 61. doi:10.1016/j.bushor.2009.09.003.
[4]boyd, danah m.; Ellison, Nicole B. (2007). "Social Network Sites: Definition, History, and Scholarship". Journal of Computer-Mediated Communication. 13 (1): 210–30. 
     
Ref:《Who interacts on the Web?: The intersection of users’ personality and social media use》
%In the increasingly user-generated Web, users’ personality traits may be crucial factors leading them to engage in this participatory media. The literature suggests factors such as extraversion, emotional stability and openness to experience are related to uses of social applications on the Internet. Using a national sample of US adults, this study investigated the relationship between these three dimensions of the Big-Five model and social media use (defined as use of social networking sites and instant messages). It also examined whether gender and age played a role in that dynamic. Results revealed that while extraversion and openness to experiences were positively related to social media use, emotional stability was a negative predictor, controlling for socio-demographics and life satisfaction. These findings differed by gender and age. While extraverted men and women were both likely to be more frequent users of social media tools, only the men with greater degrees of emotional instability were more regular users. The relationship between extraversion and social media use was particularly important among the young adult cohort. Conversely, being open to new experiences emerged as an important personality predictor of social media use for the more mature segment of the sample.
在越来越多的用户生成的Web中,用户的个性特征可能是导致他们参与这种参与性媒体的关键因素。文献表明,诸如外向性,情绪稳定性和对体验的开放性等因素与互联网上社交应用的使用有关。本研究使用美国成年人的全国样本,调查了大五人格模型和社交媒体使用(定义为使用社交网站和即时消息)这三个维度之间的关系。它还研究了性别和年龄是否在这种动态中发挥了作用。结果显示,虽然外向和对经验的开放与社交媒体使用呈正相关,但情绪稳定性是一个负面预测因素,可以控制社会人口统计和生活满意度。这些发现因性别和年龄而异。虽然外向的男性和女性都可能更频繁地使用社交媒体工具,但只有情绪不稳定程度较高的男性才更常见。外向性和社交媒体使用之间的关系在年轻成年人群中尤为重要。相反,对于更成熟的样本部分,对新体验持开放态度是社交媒体使用的重要人格预测指标。
(大五人格:
开放性(openness):具有想象、审美、情感丰富、求异、创造、智能等特质。
责任心(conscientiousness):显示胜任、公正、条理、尽职、成就、自律、谨慎、克制等特点。
外倾性(extraversion):表现出热情、社交、果断、活跃、冒险、乐观等特质。
宜人性(agreeableness):具有信任、利他、直率、依从、谦虚、移情等特质。
神经质或情绪稳定性(neuroticism):具有平衡焦虑、敌对、压抑、自我意识、冲动、脆弱等情绪的特质,即具有保持情绪稳定的能力
)



\section{哪些方面与儿童青少年认知发生联系,结合到认知控制的影响是怎样的  %\textcolor{red}{
%(兰于权)}
}

\section{新的值得关注的现象:社交成瘾?  % \textcolor{red}{
%(兰于权、宋伊一)}
}

Online Social Networking and Addiction—A Review of the Psychological Literature
Daria J. Kuss *   andMark D. Griffiths 
International Gaming Research Unit, Psychology Division, Nottingham Trent University, NG1 4BU, UK

%Author to whom correspondence should be addressed.
%Received: 9 March 2011; in revised form: 12 August 2011 / Accepted: 22 August 2011 / Published: 29 August 2011
%Abstract: Social Networking Sites (SNSs) are virtual communities where users can create individual public profiles, interact with real-life friends, and meet other people based on shared interests. They are seen as a ‘global consumer phenomenon’ with an exponential rise in usage within the last few years. Anecdotal case study evidence suggests that ‘addiction’ to social networks on the Internet may be a potential mental health problem for some users. However, the contemporary scientific literature addressing the addictive qualities of social networks on the Internet is scarce. Therefore, this literature review is intended to provide empirical and conceptual insight into the emerging phenomenon of addiction to SNSs by: (1) outlining SNS usage patterns, (2) examining motivations for SNS usage, (3) examining personalities of SNS users, (4) examining negative consequences of SNS usage, (5) exploring potential SNS addiction, and (6) exploring SNS addiction specificity and comorbidity. The findings indicate that SNSs are predominantly used for social purposes, mostly related to the maintenance of established offline networks. Moreover, extraverts appear to use social networking sites for social enhancement, whereas introverts use it for social compensation, each of which appears to be related to greater usage, as does low conscientiousness and high narcissism. Negative correlates of SNS usage include the decrease in real life social community participation and academic achievement, as well as relationship problems, each of which may be indicative of potential addiction.
%Keywords:
% social network addiction; social networking sites; literature review; motivations; personality; negative consequences; comorbidity; specificity
%1. Introduction
%“I’m an addict. I just get lost in Facebook” replies a young mother when asked why she does not see herself able to help her daughter with her homework. Instead of supporting her child, she spends her time chatting and browsing the social networking site [1]. This case, while extreme, is suggestive of a potential new mental health problem that emerges as Internet social networks proliferate. Newspaper stories have also reported similar cases, suggesting that the popular press was early to discern the potentially addictive qualities of social networking sites (SNS; i.e., [2,3]). Such media coverage has alleged that women are at greater risk than men for developing addictions to SNSs [4].
%The mass appeal of social networks on the Internet could potentially be a cause for concern, particularly when attending to the gradually increasing amounts of time people spend online [5]. On the Internet, people engage in a variety of activities some of which may be potentially to be addictive. Rather than becoming addicted to the medium per se, some users may develop an addiction to specific activities they carry out online [6]. Specifically, Young [7] argues that there are five different types of internet addiction, namely computer addiction (i.e., computer game addiction), information overload (i.e., web surfing addiction), net compulsions (i.e., online gambling or online shopping addiction), cybersexual addiction (i.e., online pornography or online sex addiction), and cyber-relationship addiction (i.e., an addiction to online relationships). SNS addiction appears to fall in the last category since the purpose and main motivation to use SNSs is to establish and maintain both on- and offline relationships (for a more detailed discussion of this please refer to the section on motivations for SNS usage). From a clinical psychologist’s perspective, it may be plausible to speak specifically of ‘Facebook Addiction Disorder’ (or more generally ‘SNS Addiction Disorder’) because addiction criteria, such as neglect of personal life, mental preoccupation, escapism, mood modifying experiences, tolerance, and concealing the addictive behavior, appear to be present in some people who use SNSs excessively [8].
%Social Networking Sites are virtual communities where users can create individual public profiles, interact with real-life friends, and meet other people based on shared interests. SNSs are “web-based services that allow individuals to: (1) construct a public or semi-public profile within a bounded system, (2) articulate a list of other users with whom they share a connection, and (3) view and traverse their list of connections and those made by others within the system” [9]. The focus is placed on established networks, rather than on networking, which implies the construction of new networks. SNSs offer individuals the possibilities of networking and sharing media content, therefore embracing the main Web 2.0 attributes [10], against the framework of their respective structural characteristics.
%In terms of SNS history, the first social networking site (SixDegrees) was launched in 1997, based on the idea that everybody is linked with everybody else via six degrees of separation [9], and initially referred to as the “small world problem” [11]. In 2004, the most successful current SNS, Facebook, was established as a closed virtual community for Harvard students. The site expanded very quickly and Facebookcurrently has more than 500 million users, of whom fifty percent log on to it every day. Furthermore, the overall time spent on Facebook increased by 566% from 2007 to 2008 [12]. This statistic alone indicates the exponential appeal of SNSs and also suggests a reason for a rise in potential SNS addiction. Hypothetically, the appeal of SNSs may be traced back to its reflection of today’s individualist culture. Unlike traditional virtual communities that emerged during the 1990s based on shared interests of their members [13], social networking sites are egocentric sites. It is the individual rather than the community that is the focus of attention [9].
%Egocentrism has been linked to Internet addiction [14]. Supposedly, the egocentric construction of SNSs may facilitate the engagement in addictive behaviors and may thus serve as a factor that attracts people to using it in a potentially excessive way. This hypothesis is in line with the PACE Framework for the etiology of addiction specificity [15]. Attraction is one of the four key components that may predispose individuals to becoming addicted to specific behaviors or substances rather than specific others. Accordingly, due to their egocentric construction, SNSs allow individuals to present themselves positively that may “raise their spirits” (i.e., enhance their mood state) because it is experienced as pleasurable. This may lead to positive experiences that can potentially cultivate and facilitate learning experiences that drive the development of SNS addiction.
%A behavioral addiction such as SNS addiction may thus be seen from a biopsychosocial perspective [16]. Just like substance-related addictions, SNS addiction incorporates the experience of the ‘classic’ addiction symptoms, namely mood modification (i.e., engagement in SNSs leads to a favourable change in emotional states), salience (i.e., behavioral, cognitive, and emotional preoccupation with the SNS usage), tolerance (i.e., ever increasing use of SNSs over time), withdrawal symptoms (i.e., experiencing unpleasant physical and emotional symptoms when SNS use is restricted or stopped), conflict (i.e., interpersonal and intrapsychic problems ensue because of SNS usage), and relapse (i.e., addicts quickly revert back in their excessive SNS usage after an abstinence period).
%Moreover, scholars have suggested that a combination of biological, psychological and social factors contributes to the etiology of addictions [16,17], that may also hold true for SNS addiction. From this it follows that SNS addiction shares a common underlying etiological framework with other substance-related and behavioral addictions. However, due to the fact that the engagement in SNSs is different in terms of the actual expression of (Internet) addiction (i.e., pathological use of social networking sites rather than other Internet applications), the phenomenon appears worthy of individual consideration, particularly when considering the potentially detrimental effects of both substance-related and behavioral addictions on individuals who experience a variety of negative consequences because of their addiction [18].
%To date, the scientific literature addressing the addictive qualities of social networks on the Internet is scarce. Therefore, with this literature review, it is intended to provide empirical insight into the emerging phenomenon of Internet social network usage and potential addiction by (1) outlining SNS usage patterns, (2) examining motivations for SNS usage, (3) examining personalities of SNS users, (4) examining negative consequences of SNSs, (5) exploring potential SNS addiction, and (6) exploring SNS addiction specificity and comorbidity.
%2. Method
%An extensive literature search was conducted using the academic database Web of Knowledge as well as Google Scholar. The following search terms as well as their derivatives were entered: social network, online network, addiction, compulsive, excessive, use, abuse, motivation, personality, and comorbidity. Studies were included if they: (i) included empirical data, (ii) made reference to usage patterns, (iii) motivations for usage, (iv) personality traits of users, (v) negative consequences of use, (vi) addiction, (vii) and/or comorbidity and specificity. A total of 43 empirical studies were identified from the literature, five of which specifically assessed SNS addiction.
%3. Results
%3.1. Usage
%Social networking sites are seen as a ‘global consumer phenomenon’ and, as already noted, have experienced an exponential rise in usage within the last few years [12]. Of all Internet users, approximately one-third participate in SNSs and ten percent of the total time spent online is spent on SNSs [12]. In terms of usage, the results of the Parents and Teens 2006 Survey with a random sample of 935 participants in America revealed that 55% of youths used SNSs in that year [19]. The main reasons reported for this usage were staying in touch with friends (endorsed by 91%), and using them to make new friends (49%). This was more common among boys than girls. Girls preferred to use these sites in order to maintain contacts with actual friends rather than making new ones. Furthermore, half of the teenagers in this sample visited their SNS at least once a day which is indicative of the fact that in order to keep an attractive profile, frequent visits are necessary and this is a factor that facilitates potential excessive use [19]. Moreover, based on the results of consumer research, the overall usage of SNSs increased by two hours per month to 5.5 hours and active participation increased by 30% from 2009 to 2010 [5].
%The findings of an online survey of 131 psychology students in the US [20] indicated that 78% used SNSs, and that 82% of males and 75% of females had SNS profiles. Of those, 57% used their SNS on a daily basis. The activities most often engaged in on SNSs were reading/responding to comments on their SNS page and/or posts to one’s wall (endorsed by 60%; the “wall” is a special profile feature in Facebook, where people can post comments, pictures, and links, that can be responded to), sending/responding to messages/invites (14%), and browsing friends’ profiles/walls/pages (13%; [20]). These results correspond with findings from a different study including another university student sample [21].
%Empirical research has also suggested gender differences in SNS usage patterns. Some studies claim that men tend to have more friends on SNSs than women [22], whereas others have found the opposite [23]. In addition, men were found to take more risks with regards to disclosure of personal information [24,25]. Furthermore, one study reported that slightly more females used MySpace specifically (i.e., 55% compared to 45% of males) [26].
%Usage of SNSs has also been found to differ with regards to age group. A study comparing 50 teenagers (13–19 years) and the same number of older MySpace users (60 years and above) revealed that teenagers’ friends’ networks were larger and that their friends were more similar to themselves with regards to age [23]. Furthermore, older users’ networks were smaller and more dispersed age-wise. Additionally, teenagers made more use of MySpace web 2.0 features (i.e., sharing video and music, and blogging) relative to older people [23].
%With regards to how people react to using SNSs, a recent study [27] using psychophysiological measures (skin conductance and facial electromyography) found that social searching (i.e., extracting information from friends’ profiles), was more pleasurable than social browsing (i.e., passively reading newsfeeds) [27]. This finding indicates that the goal-directed activity of social searching may activate the appetitive system, which is related to pleasurable experience, relative to the aversive system [28]. On a neuroanatomical level, the appetitive system has been found to be activated in Internet game overusers and addicts [29,30], which may be linked back to a genetic deficiency in the addicts’ neurochemical reward system [31]. Therefore, the activation of the appetitive system in social network users who engage in social searching concurs with the activation of that system in people found to suffer from behavioral addictions. In order to establish this link for SNS specifically, further neurobiological research is required.
%In reviewing SNS usage patterns, the findings of both consumer research and empirical research indicate that overall, regular SNS use has increased substantially over the last few years. This supports the availability hypothesis that where there is increased access and opportunity to engage in an activity (in this case SNSs), there is an increase in the numbers of people who engage in the activity [32]. Moreover, it indicates that individuals become progressively aware of this available supply and become more sophisticated with regards to their usage skills. These factors are associated with the pragmatics factor of addiction specificity etiology [15]. Pragmatics is one of the four key components of the addiction specificity model and it emphasizes access and habituation variables in the development of specific addictions. Therefore, the pragmatics of SNS usage appears to be a factor related to potential SNS addiction.
%In addition to this, the findings of the presented studies indicate that compared to the general population, teenagers and students make most use of SNSs by utilizing the inherent Web 2.0 features. Additionally, there appear to be gender differences in usage, the specifics of which are only vaguely defined and thus require further empirical investigation. In addition, SNSs tend to be used mostly for social purposes of which extracting further information from friends’ pages appears particularly pleasurable. This, in turn, may be linked to the activation of the appetitive system, which indicates that engaging in this particular activity may stimulate the neurological pathways known to be related to addiction experience.
%3.2. Motivations
%Studies suggest that SNS usage in general, and Facebook in particular, differs as a function of motivation (i.e., [33]). Drawing on uses and gratification theory, media are used in a goal-directed way for the purpose of gratification and need satisfaction [34] which have similarities with addiction. Therefore, it is essential to understand the motivations that underlie SNS usage. Persons with higher social identity (i.e., solidarity to and conformity with their own social group), higher altruism (related to both, kin and reciprocal altruism) and higher telepresence (i.e., feeling present in the virtual environment) tend to use SNSs because they perceive encouragement for participation from the social network [35]. Similarly, the results of a survey comprising 170 US university students indicated that social factors were more important motivations for SNS usage than individual factors [36]. More specifically, these participants’ interdependent self-construal (i.e., the endorsement of collectivist cultural values), led to SNS usage that in turn resulted in higher levels of satisfaction, relative to independent self-construal, which refers to the adoption of individualist values. The latter were not related to motivations for using SNSs [36].
%Another study by Barker [37] presented similar results, and found that collective self-esteem and group identification positively correlated with peer group communication via SNSs. Cheung, Chiu and Lee [38] assessed social presence (i.e., the recognition that other persons share the same virtual realm, the endorsement of group norms, maintaining interpersonal interconnectivity and social enhancement with regards to SNS usage motivations). More specifically, they investigated the We-intention to use Facebook (i.e., the decision to continue using a SNS together in the future). The results of their study indicated that We-intention positively correlated with the other variables [38].
%Similarly, social reasons appeared as the most important motives for using SNSs in another study [20]. The following motivations were endorsed by the participating university student sample: keeping in touch with friends they do not see often (81%), using them because all their friends had accounts (61%), keeping in touch with relatives and family (48%), and making plans with friends they see often (35%). A further study found that a large majority of students used SNSs for the maintenance of offline relationships, whereas some preferred to use this type of Internet application for communication rather than face-to-face interaction [39].
%The particular forms of virtual communication in SNSs include both asynchronous (i.e., personal messages sent within the SNS) and synchronous modes (i.e., embedded chat functions within the SNS) [40]. On behalf of the users, these communication modes require learning differential vocabularies, namely Internet language [41,42]. The idiosyncratic form of communication via SNSs is another factor that may fuel potential SNS addiction because communication has been identified as a component of the addiction specificity etiology framework [15]. Therefore, it can be hypothesized that users who prefer communication via SNSs (as compared to face-to-face communication) are more likely to develop an addiction to using SNSs. However, further empirical research is needed to confirm such a speculation.
%Moreover, research suggests that SNSs are used for the formation and maintenance of different forms of social capital [43]. Social capital is broadly defined as “the sum of the resources, actual or virtual, that accrue to an individual or a group by virtue of possessing a durable network of more or less institutionalized relationships of mutual acquaintance and recognition” [44]. Putnam [45] differentiates bridging and bonding social capital from one another. Bridging social capital refers to weak connections between people that are based on information-sharing rather than emotional support. These ties are beneficial in that they offer a wide range of opportunities and access to broad knowledge because of the heterogeneity of the respective network’s members [46]. Alternatively, bonding social capital indicates strong ties usually between family members and close friends [45].
%SNSs are thought to increase the size of potential networks because of the large number of possible weak social ties among members, which is enabled via the structural characteristics of digital technology [47]. Therefore, SNSs do not function as communities in the traditional sense. They do not include membership, shared influence, and an equal power allocation. Instead, they can be conceptualized as networked individualism, allowing the establishment of numerous self-perpetuating connections that appear advantageous for users [48]. This is supported by research that was carried out on a sample of undergraduate students [43]. More specifically, this study found that maintaining bridging social capital via participation in SNSs appeared to be beneficial for students with regards to potential employment opportunities in addition to sustaining ties with old friends. Overall, the benefits of bridging social capital formed via participation in SNSs appeared to be particularly advantageous for individuals with low-self esteem [49]. However, the ease of establishing and maintaining bridging social capital may become one of the reasons why people with low self-esteem are drawn to using SNSs in a potentially excessive manner. Lower self-esteem, in turn, has been linked to Internet addiction [50,51].
%Furthermore, SNS usage has been found to differ between people and cultures. A recent study [52] including samples from the US, Korea and China demonstrated that the usage of different Facebook functions was associated with the creation and maintenance of either bridging or bonding social capital. People in the US used the ‘Communication’ function (i.e., conversation and opinion sharing) in order to bond with their peers. However, Koreans and Chinese used ‘Expert Search’ (i.e., searching for associated professionals online) and ‘Connection’ (i.e., maintaining offline relationships) for the formation and sustaining of both bonding and bridging social capital [52]. These findings indicate that due to cultural differences in SNS usage patterns, it appears necessary to investigate and contrast SNS addiction in different cultures in order to discern both similarities and differences.
%Additionally, the results of an online survey with a student convenience sample of 387 participants [53] indicated that several factors significantly predicted the intention to use SNSs as well as their actual usage. The identified predictive factors were (i) playfulness (i.e., enjoyment and pleasure), (ii) the critical mass of the users who endorsed the technology, (iii) trust in the site, (iv) perceived ease of use, and (v) perceived usefulness. Moreover, normative pressure (i.e., the expectations of other people with regards to one’s behavior) had a negative relationship with SNS usage. These results suggest that it is particularly the enjoyment associated with SNS use in a hedonic context (which has some similarities to addictions), as well as the recognition that a critical mass uses SNSs that motivates people to make use of those SNSs themselves [53].
%Another study [54] used a qualitative methodology to investigate why teenagers use SNSs. Interviews were conducted with 16 adolescents aged 13 to 16 years. The results indicated that the sample used SNSs in order to express and actualize their identities either via self-display of personal information (which was true for the younger sample) or via connections (which was true for the older participants). Each of these motivations was found to necessitate a trade-off between potential opportunities for self-expression and risks with regards to compromising privacy on behalf of the teenagers [54].
%A study by Barker [37] also suggested there may be differences in motivations for SNS use between men and women. Females used SNSs for communication with peer group members, entertainment and passing time, whereas men used it in an instrumental way for social compensation, learning, and social identity gratifications (i.e., the possibility to identify with group members who share similar characteristics). Seeking friends, social support, information, and entertainment were found to be the most significant motivations for SNS usage in a sample of 589 undergraduate students [55]. In addition to this, endorsement of these motivations was found to differ across cultures. Kim et al. [55] found that Korean college students sought social support from already established relationships via SNSs, whereas American college students looked for entertainment. Similarly, Americans had significantly more online friends than Koreans, suggesting that the development and maintenance of social relationships on SNSs was influenced by cultural artefacts [55]. Furthermore, technology-relevant motivations were related to SNS use. The competence in using computer-mediated communication (i.e., the motivation to, knowledge of, and efficacy in using electronic forms of communication) was found to be significantly associated with spending more time on Facebook and checking one’s wall significantly more often [33].
%Overall, the results of these studies indicate that SNSs are predominantly used for social purposes, mostly related to the maintenance of established offline networks, relative to individual ones. In line with this, people may feel compelled to maintaining their social networks on the Internet which may lead to using SNSs excessively. The maintenance of already established offline networks itself can therefore be seen as an attraction factor, which according to Sussman et al. [15] is related to the etiology of specific addictions. Furthermore, viewed from a cultural perspective, it appears that motivations for usage differ between members of Asian and Western countries as well as between genders and age groups. However, in general, the results of the reported studies suggest that the manifold ties pursued online are indicative, for the most part, of bridging rather than bonding social capital. This appears to show that SNSs are primarily used as a tool for staying connected.
%Staying connected is beneficial to such individuals because it offers them a variety of potential academic and professional opportunities, as well as access to a large knowledge base. As the users’ expectations of connectivity are met through their SNS usage, the potential for developing SNS addiction may increase as a consequence. This is in accordance with the expectation factor that drives the etiology of addiction to a specific behavior [15]. Accordingly, the supposed expectations and benefits of SNS use may go awry particularly for people with low self-esteem. They may feel encouraged to spend excessive amounts of time on SNSs because they perceive it as advantageous. This, in turn, may potentially develop into an addiction to using SNSs. Clearly, future research is necessary in order to establish this link empirically.
%Moreover, there appear certain limitations to the studies presented. Many studies included small convenience samples, teenagers or university students as participants, therefore severely limiting the generalizability of findings. Thus, researchers are advised to take this into consideration and amend their sampling frameworks by using more representative samples and thus improve the external validity of the research.
%3.3. Personality
%A number of personality traits appear to be associated with the extent of SNS use. The findings of some studies (e.g., [33,56]) indicate that people with large offline social networks, who are more extroverted, and who have higher self-esteem, useFacebook for social enhancement, supporting the principle of ‘the rich get richer’. Correspondingly, the size of people’s online social networks correlates positively with life satisfaction and well-being [57], but does neither have an effect on the size of the offline network nor on emotional closeness to people in real life networks [58].
%However, people with only a few offline contacts compensate for their introversion, low-self esteem, and low life-satisfaction by using Facebook for online popularity, thus corroborating the principle of ‘the poor get richer’ (i.e., the social compensation hypothesis) [37,43,56,59]. Likewise, people higher in narcissistic personality traits tend to be more active on Facebook and other SNSs in order to present themselves favourably online because the virtual environment empowers them to construct their ideal selves [59–62]. The relationship between narcissism and Facebook activity may be related to the fact that narcissists have an imbalanced sense of self, fluctuating between grandiosity with regards to explicit agency and low self-esteem concerning implicit communion and vulnerability [63,64]. Narcissistic personality, in turn, has been found to be associated with addiction [65]. This finding will be discussed in more detail in the section on addiction.
%Moreover, it appears that people with different personality traits differ in their usage of SNSs [66] and prefer to use distinct functions of Facebook [33]. People high in extraversion and openness to experience use SNSs more frequently, with the former being true for mature and the latter for young people [66]. Furthermore, extraverts and people open to experiences are members of significantly more groups on Facebook,use socializing functions more [33], and have more Facebook friends than introverts [67], which delineates the former’s higher sociability in general [68]. Introverts, on the other hand, disclose more personal information on their pages [67]. Additionally, it appears that particularly shy people spend large amounts of time on Facebook and have large amounts of friends on this SNS [69]. Therefore, SNSs may appear beneficial for those whose real-life networks are limited because of the possibility of easy access to peers without the demands of real-life proximity and intimacy. This ease of access entails a higher time commitment for this group, which may possibly result in excessive and/or potentially addictive use.
%Likewise, men with neurotic traits use SNSs more frequently than women with neurotic traits [66]. Furthermore, neurotics (in general) tend to use Facebook’s wall function, where they can receive and post comments, whereas people with low neuroticism scores prefer posting photos [33]. This may be due to the neurotic individual’s greater control over emotional content with regards to text-based posts rather than visual displays [33]. However, another study [67] found the opposite, namely that people scoring high on neuroticism were more inclined to post their photographs on their page. In general, the findings for neuroticism imply that those scoring high on this trait disclose information because they seek self-assurance online, whereas those scoring low are emotionally secure and thus share information in order to express themselves [67]. High self-disclosure on SNSs, in turn, was found to positively correlate with measures of subjective well-being [57]. It remains questionable whether this implies that low self-disclosure on SNSs may be related to higher risk for potential addiction. By disclosing more personal information on their pages, users put themselves at risk for negative feedback, which has been linked to lower well-being [70]. Therefore, the association between self-disclosure on SNSs and addiction needs to be addressed empirically in future studies.
%With regards to agreeableness, it was found that females scoring high on this trait upload significantly more pictures than females scoring low, with the opposite being true for males [67]. In addition to this, people with high conscientiousness were found to have significantly more friends and to upload significantly less pictures than those scoring low on this personality trait [67]. An explanation for this finding may be that conscientious people tend to cultivate their online and offline contacts more without the necessity to share too much personal information publicly.
%Overall, the results of these studies suggest that extraverts use SNSs for social enhancement, whereas introverts use it for social compensation, each of which appears to be related to greater SNS usage. With regards to addiction, both groups could potentially develop addictive tendencies for different reasons, namely social enhancement and social compensation. In addition, the dissimilar findings of studies with regards to the number of friends introverts have online deserve closer scrutiny in future research. The same applies for the results with regards to neuroticism. On the one hand, neurotics use SNSs frequently. On the other hand, studies indicate different usage preferences for people who score high on neuroticism, which calls for further investigation. Furthermore, the structural characteristics of these Internet applications, (i.e., their egocentric construction) appear to allow favourable self-disclosure, which draws narcissists to use it. Finally, agreeableness and conscientiousness appear to be related to the extent of SNS usage. Higher usage associated with narcissistic, neurotic, extravert and introvert personality characteristics may implicate that each of these groups is particularly at risk for developing an addiction to using SNSs.
%3.4. Negative Correlates
%Some studies have highlighted a number of potential negative correlates of extensive SNS usage. For instance, the results of an online survey of 184 Internet users indicated that people who use SNS more in terms of time spent on usage were perceived to be less involved with their real life communities [71]. This is similar to the finding that people who do not feel secure about their real-life connections to peers and thus have a negative social identity tend to use SNSs more in order to compensate for this [37]. Moreover, it seems that the nature of the feedback from peers that is received on a person’s SNS profile determines the effects of SNS usage on wellbeing and self-esteem.
%More specifically, Dutch adolescents aged 10 to 19 years who received predominantly negative feedback had low self-esteem which in turn led to low wellbeing [70]. Given that people tend to be disinhibited when they are online [72], giving and receiving negative feedback may be more common on the Internet than in real life. This may entail negative consequences particularly for people with low self-esteem who tend to use SNSs as compensation for real-life social network paucity because they are dependent upon the feedback they receive via these sites [43]. Therefore, potentially, people with lower self-esteem are a population at risk for developing an addiction to using SNSs.
%According to a more recent study assessing the relationships between Facebookusage and academic performance in a sample of 219 university students [73], Facebook users had lower Grade Point Averages and spent less time studying than students who did not use this SNS. Of the 26% of students reporting an impact of their usage on their lives, three-quarters (74%) claimed that it had a negative impact, namely procrastination, distraction, and poor time-management. A potential explanation for this may be that students who used the Internet to study may have been distracted by simultaneous engagement in SNSs, implying that this form of multitasking is detrimental to academic achievement [73].
%In addition to this, it appears that the usage of Facebook may in some circumstances have negative consequences for romantic relationships. The disclosure of rich private information on one’s Facebook page including status updates, comments, pictures, and new friends, can result in jealous cyberstalking [74], including interpersonal electronic surveillance (IES; [75]) by one’s partner. This was reported to lead to jealousy [76,77] and, in the most extreme cases, divorce and associated legal action [78].
%These few existent studies highlight that in some circumstances, SNS usage can lead to a variety of negative consequences that imply a potential decrease in involvement in real-life communities and worse academic performance, as well as relationship problems. Reducing and jeopardizing academic, social and recreational activities are considered as criteria for substance dependence [18] and may thus be considered as valid criteria for behavioral addictions [79], such as SNS addiction. In light of this, endorsing these criteria appears to put people at risk for developing addiction and the scientific research base outlined in the preceding paragraphs supports the potentially addictive quality of SNSs.
%Notwithstanding these findings, due to the lack of longitudinal designs used in the presented studies, no causal inferences can be drawn with regards to whether the excessive use of SNSs is the causal factor for the reported negative consequences. Moreover, potential confounders need to be taken into consideration. For instance, the aspect of university students’ multi-tasking when studying appears to be an important factor related to poor academic achievement. Moreover, pre-existent relationship difficulties in the case of romantic partners may potentially be exacerbated by SNS use, whereas the latter does not necessarily have to be the primary driving force behind the ensuing problems. Nevertheless, the findings support the idea that SNSs are used by some people in order to cope with negative life events. Coping, in turn, has been found to be associated with both substance dependence and behavioral addictions [80]. Therefore, it appears valid to claim that there is a link between dysfunctional coping (i.e., escapism and avoidance) and excessive SNS use/addiction. In order to substantiate this conjecture and to more fully investigate the potential negative correlates associated with SNS usage, further research is needed.
%3.5. Addiction
%Researchers have suggested that the excessive use of new technologies (and especially online social networking) may be particularly addictive to young people [81]. In accordance with the biopsychosocial framework for the etiology of addictions [16] and the syndrome model of addiction [17], it is claimed that those people addicted to using SNSs experience symptoms similar to those experienced by those who suffer from addictions to substances or other behaviors [81]. This has significant implications for clinical practice because unlike other addictions, the goal of SNS addiction treatment cannot be total abstinence from using the Internet per se since the latter is an integral element of today’s professional and leisure culture. Instead, the ultimate therapy aim is controlled use of the Internet and its respective functions, particularly social networking applications, and relapse prevention using strategies developed within cognitive-behavioral therapies [81].
%In addition to this, scholars have hypothesized that young vulnerable people with narcissistic tendencies are particularly prone to engaging with SNSs in an addictive way [65]. To date, only three empirical studies have been conducted and published in peer-reviewed journals that have specifically assessed the addictive potential of SNSs [82–84]. In addition to this, two publicly available Master’s theses have analyzed the SNS addiction and will be presented subsequently for the purpose of inclusiveness and the relative lack of data on the topic [85,86]. In the first study [83], 233 undergraduate university students (64% females, mean age = 19 years, SD = 2 years) were surveyed using a prospective design in order to predict high level use intentions and actual high-level usage of SNSs via an extended model of the theory of planned behavior (TPB; [87]). High-level usage was defined as using SNSs at least four times per day. TPB variables included measures of intention for usage, attitude, subjective norm, and perceived behavioral control (PBC). Furthermore, self-identity (adapted from [88]), belongingness [89], as well as past and potential future usage of SNSs were investigated. Finally, addictive tendencies were assessed using eight questions scored on Likert scales (based on [90]).
%One week after completion of the first questionnaire, participants were asked to indicate on how many days during the last week they had visited SNSs at least four times a day. The results of this study indicated that past behavior, subjective norm, attitude, and self-identity significantly predicted both behavioral intention as well as actual behavior. Additionally, addictive tendencies with regards to SNS use were significantly predicted by self-identity and belongingness [83]. Therefore, those who identified themselves as SNS users and those who looked for a sense of belongingness on SNSs appeared to be at risk for developing an addiction to SNSs.
%In the second study [82], an Australian university student sample of 201 participants (76% female, mean age = 19, SD = 2) was drawn upon in order to assess personality factors via the short version of the NEO Personality Inventory (NEO-FFI; [91]), the Self-Esteem Inventory (SEI; [92]), time spent using SNSs, and an Addictive Tendencies Scale (based on [90,93]). The Addictive Tendencies Scale included three items measuring salience, loss of control, and withdrawal. The results of a multiple regression analysis indicated that high extraversion and low conscientiousness scores significantly predicted both addictive tendencies and the time spent using an SNS. The researchers suggested that the relationship between extraversion and addictive tendencies could be explained by the fact that using SNSs satisfies the extraverts’ need to socialize [82]. The findings with regards to lack of conscientiousness appear to be in line with previous research on the frequency of general Internet use in that people who score low on conscientiousness tend to use the Internet more frequently than those who score high on this personality trait [94].
%In the third study, Karaiskos et al. [84] report the case of a 24-year old female who used SNS to such an extent that her behavior significantly interfered with her professional and private life. As a consequence, she was referred to a psychiatric clinic. She used Facebook excessively for at least five hours a day and was dismissed from her job because she continuously checked her SNS instead of working. Even during the clinical interview, she used her mobile phone to access Facebook. In addition to excessive use that led to significant impairment in a variety of areas in the woman’s life, she developed anxiety symptoms as well as insomnia, which suggestively points to the clinical relevance of SNS addiction. Such extreme cases have led to some researchers to conceptualize SNS addiction as Internet spectrum addiction disorder [84]. This indicates that first, SNS addiction can be classified within the larger framework of Internet addictions, and second, that it is a specific Internet addiction, alongside other addictive Internet applications such as Internet gaming addiction [95], Internet gambling addiction [96], and Internet sex addiction [97].
%In the fourth study [85], SNS game addiction was assessed via the Internet Addiction Test [98] using 342 Chinese college students aged 18 to 22 years. In this study, SNS game addiction referred specifically to being addicted to the SNS gameHappy Farm. Students were defined as addicted to using this SNS game when they endorsed a minimum of five out of eight total items of the IAT. Using this cut-off, 24% of the sample were identified as addicted [85].
%Moreover, the author investigated gratifications of SNS game use, loneliness [99], leisure boredom [100], and self esteem [101]. The findings indicated that there was a weak positive correlation between loneliness and SNS game addiction and a moderate positive correlation between leisure boredom and SNS game addiction. Moreover, the gratifications “inclusion” (in a social group) and “achievement” (in game), leisure boredom, and male gender significantly predicted SNS game addiction [85].
%In the fifth study [86], SNS addiction was assessed in a sample of 335 Chinese college students aged 19 to 28 years using Young’s Internet Addiction Test [98] modified to specifically assess the addiction to a common Chinese SNS, namely Xiaonei.com. Users were classified as addicted when they endorsed five or more of the eight addiction items specified in the IAT. Moreover, the author assessed loneliness [99], user gratifications (based on the results of a previous focus group interview), usage attributes and patterns of SNS website use [86].
%The results indicated that of the total sample, 34% were classified as addicted. Moreover, loneliness significantly and positively correlated with frequency and session length of using Xiaonei.com as well as SNS addiction. Likewise, social activities and relationship building were found to predict SNS addiction [86].
%Unfortunately, when viewed from a critical perspective, the quantitative studies reviewed here suffer from a variety of limitations. Initially, the mere assessment of addiction tendencies does not suffice to demarcate real pathology. In addition, the samples were small, specific, and skewed with regards to female gender. This may have led to the very high addiction prevalence rates (up to 34%) reported [86]. Clearly, it needs to be ensured that rather than assessing excessive use and/or preoccupation, addiction specifically needs to be assessed.
%Wilson et al.’s study [82] suffered from endorsing only three potential addiction criteria which is not sufficient for establishing addiction status clinically. Similarly, significant impairment and negative consequences that discriminate addiction from mere abuse [18] were not assessed in this study at all. Thus, future studies have great potential in addressing the emergent phenomenon of addiction to using social networks on the Internet by means of applying better methodological designs, including more representative samples, and using more reliable and valid addiction scales so that current gaps in empirical knowledge can be filled.
%Furthermore, research must address the presence of specific addiction symptoms beyond negative consequences. These might be adapted from the DSM-IV TR criteria for substance dependence [18] and the ICD-10 criteria for a dependence syndrome [102], including (i) tolerance, (ii) withdrawal, (iii) increased use, (iv) loss of control, (v) extended recovery periods, (vi) sacrificing social, occupational and recreational activities, and (vii) continued use despite of negative consequences. These have been found to be adequate criteria for diagnosing behavioral addictions [79] and thus appear sufficient to be applied to SNS addiction. In order to be diagnosed with SNS addiction, at least three (but preferably more) of the above mentioned criteria should be met in the same 12-month period and they must cause significant impairment to the individual [18].
%In light of this qualitative case study, it appears that from a clinical perspective, SNS addiction is a mental health problem that may require professional treatment. Unlike the quantitative studies, the case study emphasizes the significant individual impairment that is experienced by individuals that spans a variety of life domains, including their professional life as well as their psychosomatic condition. Future researchers are therefore advised to not only investigate SNS addiction in a quantitative way, but to further our understanding of this new mental health problem by analyzing cases of individuals who suffer from excessive SNS usage.
%3.6. Specificity and Comorbidity
%It appears essential to pay adequate attention to (i) the specificity of SNS addiction and (ii) potential comorbidity. Hall et al. [103] outline three reasons why it is necessary to address comorbidity between mental disorders, such as addictions. First, a large number of mental disorders feature additional (sub)clinical problems/disorders. Second, comorbid conditions must be addressed in clinical practice in order to improve treatment outcomes. Third, specific prevention programs may be developed which incorporate different dimensions and treatment modalities that particularly target associated mental health problems. From this it follows that assessing the specificity and potential comorbidities of SNS addiction is important. However, to date, research addressing this topic is virtually non-existent. There has been almost no research on the co-occurrence of SNS addiction with other types of addictive behavior, mainly because there have been so few studies examining SNS addiction as highlighted in the previous section. However, based on the small empirical base, there are a number of speculative assumptions that can be made about co-addiction co-morbidity in relation to SNS addiction.
%Firstly, for some individuals, their SNS addiction takes up such a large amount of available time that it is highly unlikely that it would co-occur with other behavioral addictions unless the other behavioral addiction(s) can find an outlet via social networking sites (e.g., gambling addiction, gaming addiction). Put simply, there would be little face validity in the same individual being, for example, both a workaholic and a social networking addict, or an exercise addict and a social networking addict, mainly because the amount of daily time available to engage in two behavioral addictions simultaneously would be highly unlikely. Still, it is necessary to pinpoint the respective addictive behaviors because some of these behaviors may in fact co-occur. In one study that included a clinical sample diagnosed with substance dependencies, Malat and colleagues [104] found that 61% pursued at least one and 31% engaged in two or more problematic behaviors, such as overeating, unhealthy relationships and excessive Internet use. Therefore, although a simultaneous addiction to behaviors such as working and using SNS is relatively unlikely, SNS addiction may potentially co-occur with overeating and other excessive sedentary behaviors.
%Thus, secondly, it is theoretically possible for a social networking addict to have an additional drug addiction, as it is perfectly feasible to engage in both a behavioral and chemical addiction simultaneously [16]. It may also make sense from a motivational perspective. For instance, if one of the primary reasons social network addicts are engaging in the behavior is because of their low self-esteem, it makes intuitive sense that some chemical addictions may serve the same purpose. Accordingly, studies suggest that the engagement in addictive behaviors is relatively common among persons who suffer from substance dependence. In one study, Black et al. [105] found that 38% of problematic computer users in their sample had a substance use disorder in addition to their behavioral problems/addiction. Apparently, research indicates that some persons who suffer from Internet addiction experience other addictions at the same time.
%Of a patient sample including 1,826 individuals treated for substance addictions (mainly cannabis addiction), 4.1% were found to suffer from Internet addiction [106]. Moreover, the findings of further research [107] indicated that Internet addiction and substance use experience in adolescents share common family factors, namely higher parent-adolescent conflict, habitual alcohol use of siblings, perceived parents’ positive attitude to adolescent substance use, and lower family functioning. Moreover, Lam et al. [108] assessed Internet addiction and associated factors in a sample of 1,392 adolescents aged 13–18 years. In terms of potential comorbidity, they found that drinking behavior was a risk factor for being diagnosed with Internet addiction using the Internet Addiction Test [109]. This implies that potentially, alcohol abuse/dependence can be associated with SNS addiction. Support for this comes from Kuntsche et al. [110]. They found that in Swiss adolescents, the expectancy of social approval was associated with problem drinking. Since SNSs are inherently social platforms that are used by people for social purposes, it appears reasonable to deduce that there may indeed be people who suffer from comorbid addictions, namely SNS addiction and alcohol dependence.
%Thirdly, it appears that there may be a relationship between SNS addiction specificity and personality traits. Ko et al. [111] found that Internet addiction (IA) was predicted by high novelty seeking (NS), high harm avoidance (HA), and low reward dependence (RD) in adolescents. Those adolescents who were addicted to the Internet and who had experience of substance use scored significantly higher on NS and lower on HA than the IA group. Therefore, it appears that HA particularly impacts Internet addiction specificity because high HA discriminates Internet addicts from individuals who are not only addicted to the Internet, but who use substances. Therefore, it seems plausible to hypothesize that persons with low harm avoidance are in danger of developing comorbid addictions to SNSs and substances. Accordingly, research needs to address this difference specifically for those who are addicted to using SNSs in order to demarcate this potential disorder from comorbid conditions.
%In addition to this, it seems reasonable to specifically address the respective activities people can engage in on their SNS. There have already been a number of researchers who have begun to examine the possible relationship between social networking and gambling [112–116], and social networking and gaming [113,116,117]. All of these writings have noted how the social networking medium can be used for gambling and/or gaming. For instance, online poker applications and online poker groups on social networking sites are among the most popular [115], and others have noted the press reports surrounding addiction to social networking games such asFarmville [117]. Although there have been no empirical studies to date examining addiction to gambling or gaming via social networking, there is no reason to suspect that those playing in the social networking medium are any less likely than those playing other online or offline media to become addicted to gambling and/or gaming.
%Synoptically, addressing the specificity of SNS addiction and comorbidities with other addictions is necessary for (i) comprehending this disorder as distinct mental health problem while (ii) paying respect to associated conditions, which will (iii) aid treatment and (iv) prevention efforts. From the reported studies, it appears that the individual’s upbringing and psychosocial context are influential factors with regards to potential comorbidity between Internet addiction and substance dependence, which is supported by scientific models of addictions and their etiology [16,17]. Moreover, alcohol and cannabis dependence were outlined as potential co-occurring problems. Nonetheless, apart from this, the presented studies do not specifically address the discrete relationships between particular substance dependencies and individual addictive behaviors, such as addiction to using SNSs. Therefore, future empirical research is needed in order to shed more light upon SNS addiction specificity and comorbidity.
%4. Discussion and Conclusions
%The aim of this literature review was to present an overview of the emergent empirical research relating to usage of and addiction to social networks on the Internet. Initially, SNSs were defined as virtual communities offering their members the possibility to make use of their inherent Web 2.0 features, namely networking and sharing media content. The history of SNSs dates back to the late 1990s, suggesting that they are not as new as they may appear in the first place. With the emergence of SNSs such as Facebook, overall SNS usage has accelerated in such a way that they are considered a global consumer phenomenon. Today, more than 500 million users are active participants in the Facebook community alone and studies suggest that between 55% and 82% of teenagers and young adults use SNSs on a regular basis. Extracting information from peers’ SNS pages is an activity that is experienced as especially enjoyable and it has been linked with the activation of the appetitive system, which in turn is related to addiction experience.
%In terms of sociodemographics, the studies presented indicate that overall, SNS usage patterns differ. Females appear to use SNS in order to communicate with members of their peer group, whereas males appear to use them for the purposes of social compensation, learning, and social identity gratifications [37]. Furthermore, men tend to disclose more personal information on SNS sites relative to women [25,118]. Also, more women were found to use MySpace specifically relative to men [26]. Moreover, usage patterns were found to differ between genders as a function of personality. Unlike women with neurotic traits, men with neurotic traits were found to be more frequent SNS users [66]. In addition to this, it was found that males were more likely to be addicted to SNS games specifically relative to females [85]. This is in line with the finding that males in general are a population at risk for developing an addiction to playing online games [95].
%The only study that assessed age differences in usage [23] indicated that the latter in fact varies as a function of age. Specifically, “silver surfers” (i.e., those over the age of 60 years) have a smaller circle of online friends that differs in age relative to younger SNS users. Based on the current empirical knowledge that has predominantly assessed young teenage and student samples, it appears unclear whether older people use SNSs excessively and whether they potentially become addicted to using them. Therefore, future research must aim at filling this gap in knowledge.
%Next, the motivations for using SNSs were reviewed on the basis of needs and gratifications theory. In general, research suggests that SNSs are used for social purposes. Overall, the maintenance of connections to offline network members was emphasized rather than the establishment of new ties. With regards to this, SNS users sustain bridging social capital through a variety of heterogeneous connections to other SNS users. This appeared to be beneficial for them with regards to sharing knowledge and potential future possibilities related to employment and related areas. In effect, the knowledge that is available to individuals via their social network can be thought of as “collective intelligence” [119].
%Collective intelligence extends the mere idea of shared knowledge because it is not restricted to knowledge shared by all members of a particular community. Instead, it denotes the aggregation of each individual member’s knowledge that can be accessed by other members of the respective community. In this regard, the pursuit of weak ties on SNSs is of great benefit and thus coincides with the satisfaction of the members’ needs. At the same time, it is experienced as gratifying. Therefore, rather than seeking emotional support, individuals make use of SNSs in order to communicate and stay in touch not only with family and friends, but also with more distant acquaintances, therefore sustaining weak ties with potentially advantageous environments. The benefits of large online social networks may potentially lead people to excessively engage in using them, which, in turn, may purport addictive behaviors.
%As regards personality psychology, certain personality traits were found to be associated with higher usage frequency that may be associated with potential abuse and/or addiction. Of those, extraversion and introversion stand out because each of these is related to more habitual participation in social networks on the Internet. However, the motivations of extraverts and introverts differ in that extraverts enhance their social networks, whereas introverts compensate for the lack of real life social networks. Presumably, the motivations for higher SNS usage of people who are agreeable and conscientious may be related to those shared by extraverts, indicating a need for staying connected and socializing with their communities. Nevertheless, of those, high extraversion was found to be related to potential addiction to using SNS, in accordance with low conscientiousness [82].
%The dissimilar motivations for usage found for members scoring high on the respective personality trait can inform future research into potential addiction to SNSs. Hypothetically, people who compensate for scarce ties with their real life communities may be at greater risk to develop addiction. In effect, in one study, addictive SNS usage was predicted by looking for a sense of belongingness in this community [83], which supports this conjecture. Presumably, the same may hold true for people who score high on neuroticism and narcissism, assuming that members of both groups tend to have low self-esteem. This supposition is informed by research indicating that people use the Internet excessively in order to cope with everyday stressors [120,121]. This may serve as a preliminary explanation for the findings regarding the negative correlates that were found to be associated with more frequent SNS usage.
%Overall, the engagement in particular activities on SNSs, such as social searching, and the personality traits that were found to be associated with greater extents of SNS usage may serve as an anchor point for future studies in terms of defining populations who are at risk for developing addiction to using social networks on the Internet. Furthermore, it is recommended that researchers assess factors that are specific to SNS addiction, including the pragmatics, attraction, communication and expectations of SNS use because these may predict the etiology of SNS addiction as based on the addiction specificity etiology framework [15]. Due to the scarcity of research in this domain with a specific focus on SNS addiction specificity and comorbidity, further empirical research is necessary. Moreover, researchers are encouraged to pay close attention to the different motivations of introverts and extraverts because each of those appears to be related to higher usage frequency. What is more, investigating the relationship of potential addiction with narcissism seems to be a fruitful area for empirical research. In addition to this, motivations for usage as well as a wider variety of negative correlates related to excessive SNS use need to be addressed.
%In addition to the above mentioned implications and suggestions for future research, specific attention needs to be paid to selecting larger samples which are representative of a broader population in order to increase the respective study’s external validity. The generalizability of results is essential in order to demarcate populations at risk for developing addiction to SNSs. Similarly, it appears necessary to conduct further psychophysiological studies in order to assess the phenomenon from a biological perspective. Furthermore, clear-cut and validated addiction criteria need to be assessed. It is insufficient to limit studies into addiction to assessing just a few criteria. The demarcation of pathology from high frequency and problematic usage necessitates adopting frameworks that have been established by the international classification manuals [18,102]. Moreover, in light of clinical evidence and practice, it appears essential to pay attention to the significant impairment that SNS addicts experience in a variety of life domains as a consequence of their abusive and/or addictive behaviors.
%Similarly, the results of data based on self-reports are not sufficient for diagnosis because research suggests that they may be inaccurate [122]. Conceivably, self-reports may be supplemented with structured clinical interviews [123] and further case study evidence as well as supplementary reports from the users’ significant others. In conclusion, social networks on the Internet are iridescent Web 2.0 phenomena that offer the potential to become part of, and make use of, collective intelligence. However, the latent mental health consequences of excessive and addictive use are yet to be explored using the most rigorous scientific methods.

1.导言
“我是个瘾君子。我只是在Facebook上迷路了“一位年轻的母亲在被问到为什么不能帮助女儿做家庭作业时回答道。她没有抚养孩子,而是把时间花在聊天和浏览社交网站上。1]这个案例虽然极端,但暗示着一个潜在的新的心理健康问题,随着互联网社交网络的激增而出现。报纸上的报道也报道了类似的案例,表明大众媒体很早就发现了社交网站潜在的令人上瘾的品质(SNS);i.e., [2,3])。这种媒体报道称,妇女比男子更容易上瘾[4].
社交网络在互联网上的广泛吸引力可能会引起人们的关注,尤其是在关注人们上网时间逐渐增加的情况下。5]在互联网上,人们从事各种各样的活动,其中一些活动可能会使人上瘾。而不是沉迷于媒体就其本身而言,一些用户可能对他们在网上进行的特定活动上瘾。6]特别是杨[7认为有五种不同类型的网络成瘾,即计算机成瘾 (i.e.,电脑游戏成瘾),信息过载 (i.e.上网成瘾),净强迫 (i.e.、网上赌博或网购成瘾),网络性成瘾(i.e.、网络色情制品或网络性上瘾),以及网络成瘾 (i.e.对网络关系上瘾)。SNS成瘾似乎属于最后一类,因为使用SNS的目的和主要动机是建立和维持在线和离线关系(更详细的讨论请参阅SNS使用动机一节)。从临床心理学家的角度来看,具体地说‘脸书成瘾障碍(或更广泛地说,是‘SNS成瘾障碍’),因为一些过度使用SNSS的人似乎存在成瘾标准,如忽视个人生活、心理专注、逃避现实、情绪改变经验、容忍和隐藏成瘾行为。8].
社交网站是虚拟社区,用户可以创建个人的公共档案,与现实生活中的朋友互动,并在共享兴趣的基础上结识其他人。SNS是“基于web的服务,允许个人:(1)在有界系统内构造公共或半公共配置文件;(2)列出与其共享连接的其他用户的列表;(3)查看和遍历系统中其他用户的连接列表”[9]重点放在已建立的网络上,而不是网络上,这意味着要建设新的网络。SNSS为个人提供了建立网络和共享媒体内容的可能性,因此采用了主要的Web2.0属性[10],与其各自结构特征的框架相对照。
就SNS的历史而言,第一个社交网站(六度)成立于1997年,是基于这样的理念,即每个人都是通过六度分离与其他人联系在一起的。9,最初称为“小世界问题”[11]在2004年,目前最成功的SNS,脸书,为哈佛学生建立了一个封闭的虚拟社区。这个网站扩张得很快,脸书目前有超过5亿用户,其中百分之五十的人每天登录。此外,花在脸书2007年至2008年增长566%[12]仅这一统计数据就表明了SNSS的指数吸引力,同时也表明了潜在SNS成瘾增加的原因。假设的话,SNS的吸引力可以追溯到它对当今个人主义文化的反映上。与上世纪90年代出现的基于成员共同利益的传统虚拟社区不同,[13],社交网站是以自我为中心的网站。关注的焦点是个人,而不是社会。9].
利己主义与网瘾有关。14]据推测,以自我为中心的SNSS结构可能会促进人们对上瘾行为的参与,因此也可能成为吸引人们以一种潜在的过度方式使用它的因素。这一假设与成瘾特异性病因学的佩斯框架相一致[15]吸引是可能使个人对特定行为或物质上瘾的四个关键因素之一,而不是特定的其他行为。因此,由于他们以自我为中心的建设,SNSS允许个人积极地展示自己,这可能会“提升他们的精神”。i.e.增强他们的情绪状态),因为这是作为愉快的体验。这可能导致积极的经验,有可能培养和促进学习经验,推动SNS成瘾的发展。
因此,可以从生物心理社会的角度来看待诸如SNS成瘾这样的行为成瘾[16]就像与物质有关的成瘾一样,SNS成瘾也包含了“经典”成瘾症状的体验,即情绪改变(i.e.,参与SNSS会使情绪状态发生有利的变化),突出(i.e.、行为、认知和情感上专注于SNS的使用),耐受性(i.e.,随着时间的推移越来越多地使用SNS),戒断症状(i.e.,当SNS的使用受到限制或停止时,会出现不愉快的身体和情绪症状),冲突(i.e.,由于使用SNS而引起的人际和内在心理问题),以及复发(i.e.成瘾者在戒断期后迅速恢复过量使用SNS)。

此外,学者们认为,生物学、心理和社会因素的综合作用导致了成瘾的病因。16,17],对SNS成瘾者来说也是如此。由此可以看出,SNS成瘾与其他物质相关和行为成瘾有着共同的病因学框架。然而,由于网络成瘾的实际表现方式不同,所以对网络成瘾的参与是不同的。i.e.这一现象似乎值得个人考虑,特别是考虑到与物质相关的成瘾和行为成瘾对因成瘾而遭受各种负面后果的个人可能产生的有害影响。18].
到目前为止,关于互联网上社交网络成瘾性的科学文献并不多见。因此,通过文献综述,本文旨在通过(1)概述SNS使用模式,(2)检查SNS用户的使用动机,(3)检查SNS用户的个性,(4)检查SNSS的负面后果,(5)探索潜在的SNS成瘾,以及(6)探讨SNS成瘾的特殊性和共病性,从而提供对网络社交网络使用和潜在成瘾现象的经验洞察。
2.方法
利用学术数据库Web进行了广泛的文献检索知识以及谷歌学者。输入了以下搜索词及其衍生物:社交网络、在线网络、成瘾、强迫症、过度使用、滥用、动机、人格和共病。如果这些研究包括:(一)包括经验数据,(二)提到使用模式,(三)使用动机,(四)使用者的个性特征,(五)使用的负面后果,(六)成瘾,(七)和(或)共病和特性。从文献中确定了总共43项实证研究,其中5项专门评估了SNS成瘾。
3.结果
3.1.使用
社交网站被视为“全球消费现象”,正如已经指出的,在过去几年里,社交网站的使用率呈指数增长。12]在所有的互联网用户中,大约有三分之一的人参加了SNSS,10%的上网时间都花在了SNSS上。12]就使用情况而言,2006年对美国935名参与者进行的家长和青少年调查结果显示,55%的年轻人在该年使用了SNSS。19]据报道,这种用法的主要原因是与朋友保持联系(91%的人表示赞同),并利用他们结交新朋友(49%)。这在男孩中比女孩中更常见。女孩们更喜欢使用这些网站,以便与真正的朋友保持联系,而不是结交新的朋友。此外,本样本中有一半的青少年每天至少访问SNS一次,这表明,为了保持有吸引力的形象,经常访问是必要的,这是一个可能导致过度使用的因素。19]此外,根据消费者调查的结果,从2009年到2010年,SNSS的总体使用时间增加了2个小时,达到5.5小时,积极参与的时间增加了30%。5].
美国131名心理学学生的在线调查结果20]表明78%的人使用SNS,82%的男性和75%的女性使用SNS。其中57%每天使用SNS。在SNS上最经常参与的活动是阅读/回复SNS页面上的评论和/或贴到墙上的帖子(得到60%的认可;“墙”是在脸书,用户可以发布评论、图片和链接,可以回复),发送/回复消息/邀请(14%),浏览朋友的个人资料/墙壁/页面(13%);[20])。这些结果与另一项研究的结果相一致,包括另一名大学生的样本[21].
实证研究也表明了SNS使用模式的性别差异。一些研究声称,男性在社交网络上的朋友往往比女性多。22,而其他人却发现了相反的[23]此外,发现男性在披露个人信息方面承担更多风险24,25]此外,一项研究报告说,更多的女性使用MySpace具体(i.e.为55%,而男性则为45%[26].
人们还发现,不同年龄组的SNS的使用也不同。一项比较50名青少年(13-19岁)和相同年龄组的研究MySpace60岁及以上的用户透露,青少年的朋友网络更大,他们的朋友在年龄上更接近自己。23]此外,老用户的网络更小,更分散的年龄。此外,青少年更多地利用MySpaceWeb2.0特性(i.e.与老年人相比,分享视频、音乐和博客[23].
关于人们对使用SNSS的反应,最近的一项研究[27]使用心理生理学测量(皮肤电导和面部肌电图)发现社交搜索(i.e.从朋友的个人资料中提取信息),比社交浏览更令人愉快(i.e.,被动地阅读新闻提要)[27]这一发现表明,社交搜索的目标导向活动相对于厌恶系统而言,可能激活了与愉悦体验相关的欲望系统。28]在神经解剖学层面上,人们发现网络游戏过度使用者和成瘾者的食欲系统被激活。29,30,这可能与成瘾者的神经化学奖励系统中的基因缺陷有关[31]因此,在从事社交搜索的社交网络用户中,欲望系统的激活与该系统在行为成瘾者身上的激活是一致的。为了特别为SNS建立这一联系,需要进一步的神经生物学研究。
在回顾SNS的使用模式时,消费者研究和实证研究的结果都表明,总的来说,SNS的常规使用在过去几年中有了很大的增加。这支持了一种假设,即如果有更多的机会和机会参与一项活动(在这种情况下,SNSS),那么从事这项活动的人数就会增加[32]此外,它还表明,个人逐渐意识到这一可用的供应,并在使用技能方面变得更加成熟。这些因素与成瘾特异性病因学的语用因素有关[15]语用学是成瘾特异性模型的四个关键组成部分之一,它强调在特定成瘾发展过程中的获取和习惯变量。因此,SNS使用的语用学似乎是与潜在SNS成瘾相关的一个因素。
此外,本文的研究结果表明,与普通人群相比,青少年和学生利用其固有的Web2.0特性使用最多。此外,在用法上似乎存在性别差异,其具体内容只是模糊界定,因此需要进一步的实证研究。此外,SNSS通常被用于社交目的,从朋友的页面中提取更多信息似乎特别令人愉快。反过来,这可能与食欲系统的激活有关,这表明参与这一特定活动可能刺激已知与成瘾经验有关的神经通路。
3.2.动机
研究表明,SNS的普遍使用,以及脸书特别是,由于动机的不同(i.e., [33])。利用使用和满足理论,媒体是以目标导向的方式来达到满足和满足的目的。34与成瘾有相似之处。因此,理解SNS使用的动机是非常重要的。具有较高社会身份的人(i.e.、与自己的社会群体团结一致)、更高的利他主义(与亲属和互惠利他主义相关)和更高的远程存在(i.e.,感觉存在于虚拟环境中)倾向于使用sns,因为他们感觉到来自社交网络的参与的鼓励。35]同样,一项由170名美国大学生组成的调查结果表明,社交因素比个人因素更重要的是使用社交网络的动机。36]更具体地说,这些参与者相互依赖的自我理解(i.e.对集体主义文化价值观的认可),导致SNS的使用,这反过来又导致了相对于独立自我理解的更高的满意度,这是指采用个人主义的价值观。后者与使用SNSS的动机无关[36].
巴克的另一项研究[37]提出了相似的结果,并发现集体自尊和群体认同与同伴群体通过SNSS的交流呈显著正相关。张、招、李[38评估的社会存在(i.e.认识到其他人拥有相同的虚拟领域,认可群体规范,保持人际间的相互联系,并在使用SNS的动机方面提高社会水平)。更具体地说,他们调查了我们打算使用的脸书 (i.e.未来继续使用SNS的决定)。他们的研究结果表明,我们的意向与其他变量呈正相关。38].
同样,在另一项研究中,社会原因似乎是使用SNSS的最重要动机。20]以下动机被参与的大学生样本所认可:与他们经常见不到的朋友保持联系(81%),因为他们所有的朋友都有账户(61%),与亲戚和家人保持联系(48%),以及与他们经常见面的朋友制定计划(35%)。一项进一步的研究发现,大多数学生使用sns来维持离线关系,而有些学生则更喜欢使用这种类型的internet应用程序来进行交流,而不是面对面的互动。39].
SNS中的特定虚拟通信形式包括异步(i.e.,在SNS中发送的个人消息)和同步模式(i.e.,在SNS中嵌入聊天功能)[40]这些交流方式代表用户需要学习不同的词汇,即互联网语言[41,42]通过SNSS进行交流的特殊形式是可能助长SNS成瘾的另一个因素,因为沟通已被确定为成瘾特异性病因框架的一个组成部分。15]因此,可以假设更喜欢通过SNSS进行通信的用户(与面对面交流相比)更有可能产生使用SNSS的成瘾。然而,需要进一步的实证研究来证实这种猜测。
此外,研究表明,SNSS用于形成和维持不同形式的社会资本[43]社会资本被广义地定义为实际或虚拟的资源的总和,这些资源是个人或群体由于拥有或多或少制度化的相互认识和承认关系的持久网络而积累起来的 [44]普特南[45]区分社会资本之间的联系和联系。沟通社会资本是指人与人之间基于信息共享而不是情感支持的薄弱联系。这些联系是有益的,因为它们提供了广泛的机会和获得广泛知识的机会,因为各网络成员的多样性[46]另外,社会资本的结合也意味着家庭成员和亲密朋友之间的紧密联系。45].
由于成员之间可能存在大量薄弱的社会关系,因此被认为增加了潜在网络的规模,这是通过数字技术的结构特点来实现的。47]因此,国家统计局并不像传统意义上的社区那样发挥作用。它们不包括成员、共同的影响力和平等的权力分配。相反,它们可以被概念化为网络个人主义,允许建立许多似乎对用户有利的自我延续的连接[48]这项研究得到了对一个本科生样本进行的研究的支持。43]更具体地说,这项研究发现,通过参加国家统计局来保持社会资本的桥梁,除了与老朋友保持联系外,对学生在潜在就业机会方面似乎也是有益的。总体而言,通过参与社会保障体系而形成的社会资本桥梁的好处,似乎对自尊心较低的个人特别有利。49]然而,建立和维持社会资本的便捷性可能会成为自尊心低下的人被吸引到以一种潜在的过度方式使用sns的原因之一。自尊心降低,反过来又与网瘾有关。50,51].
此外,SNS的使用被发现不同的人和文化。最近的研究[52]包括来自美国、韩国和中国的样本,表明使用不同的脸书职能与社会资本的建立和维持有关。在美国,人们使用“沟通”功能(i.e.交流和分享意见)以便与同龄人建立联系。然而,韩国人和中国人使用“专家搜索”(i.e.,在网上搜索相关专业人员)和“联系”(i.e.维持离线关系),以建立和维持社会资本[52]这些发现表明,由于SNS使用模式的文化差异,有必要对不同文化中的SNS成瘾现象进行调查和对比,以区分两者的异同。
此外,一项由387名参与者组成的学生方便样本在线调查的结果[53]指出有几个因素显着地预测了使用SNSS的意图及其实际使用情况。已确定的预测因素是:(I)嬉戏(i.e.(2)认可该技术的用户的临界数量,(3)对网站的信任,(4)被认为易于使用,(5)被认为有用。此外,规范压力(i.e.其他人对自己行为的期望)与SNS的使用呈负相关。这些结果表明,特别是与在享乐主义中使用SNS有关的享受(这与成瘾有一些相似之处),以及认识到临界群体使用SNS激励人们使用这些SNSS本身[53].
另一项研究[54使用定性方法调查青少年使用SNS的原因。对16名13至16岁的青少年进行了访谈。结果表明,样本使用SNS来表达和实现他们的身份,或者通过个人信息的自我显示(对于年轻的样本是这样),或者通过连接(对于年长的参与者来说是这样)。这些动机中的每一个都需要在潜在的自我表达机会和为青少年牺牲隐私的风险之间进行权衡。54].
巴克的研究[37]还建议男女在使用SNS的动机上可能存在差异。女性使用SNS与同龄人交流、娱乐和消磨时间,而男性则以一种工具的方式进行社会补偿、学习和社会认同满足感(i.e.与具有相似特征的群体成员认同的可能性)。在589名本科生中,寻求朋友、社会支持、信息和娱乐是使用社交网络的最重要动机。55]除此之外,对这些动机的认可也因文化而异。金姆等人 [55]发现韩国大学生通过社交网络寻求社会支持,而美国大学生则寻求娱乐。同样,美国人的网上朋友也比韩国人多得多,这表明网络社交关系的发展和维持受到文化制品的影响。55]此外,与技术相关的动机与SNS的使用有关。运用电脑沟通的能力(i.e.,使用电子通讯形式的动机、知识和效能被发现与花更多的时间在脸书更经常检查墙壁33].
总体而言,这些研究的结果表明,国家统计系统主要用于社会目的,相对于个人网络而言,主要是与维持已建立的离线网络有关。与此相一致,人们可能会感到不得不在互联网上维护他们的社交网络,这可能导致过度使用SNSS。因此,维护已经建立的离线网络本身就可以被看作是一个吸引因素,苏斯曼认为这是一个吸引因素。等人 [15与特定成瘾的病因有关。此外,从文化角度看,亚洲和西方国家成员之间以及性别和年龄组之间的使用动机似乎各不相同。然而,总的来说,所报告的研究结果表明,网上寻求的多种联系在很大程度上表明了社会资本的桥梁作用而不是纽带作用。这似乎表明,SNSS主要用作保持联系的工具。
保持联系对这些人是有益的,因为它为他们提供了各种潜在的学术和专业机会,以及获得大量知识库的机会。随着用户通过使用SNS来满足对连接性的期望,发展SNS成瘾的潜力可能因此而增加。这与导致上瘾的原因是特定行为的预期因素是一致的。15]因此,假设使用SNS的期望和好处可能会出错,特别是对于自尊心较低的人。他们可能会感到被鼓励将过多的时间花在SNSS上,因为他们认为这是有利的。反过来,这可能会发展成一种使用SNSS的上瘾。显然,未来的研究是必要的,以建立这一联系的经验。
此外,所提出的研究也有一定的局限性。许多研究包括小样本,青少年或大学生作为参与者,因此严重限制了研究结果的普遍性。因此,研究者应该考虑到这一点,并通过使用更有代表性的样本来修正他们的抽样框架,从而提高研究的外部有效性。
3.3.人格
一些人格特征似乎与使用SNS的程度有关。一些研究的结果(例如,[33,56)表示拥有大型离线社交网络的人,如果比较外向,并且有较高的自尊,就会使用。脸书为了提高社会水平,支持“富人变得更富”的原则。相应地,人们在线社交网络的规模与生活满意度和幸福感呈正相关。57,但既不影响离线网络的规模,也不影响现实网络中人们的情感亲密[58].
然而,只有几个离线接触的人通过以下方式来弥补他们的内向、自卑和生活满意度低下。脸书为了在网上流行,从而证实了“穷人越富”的原则。i.e.,社会补偿假说)[37,43,56,59]同样,自恋人格特质较高的人更倾向于积极参与。脸书为了在网上展示自己,虚拟环境赋予了他们构建理想自我的能力。59–62]自恋与脸书活动可能与自恋者有不平衡的自我意识有关,在外显的行为上的浮夸和内隐交流与脆弱的自尊心之间波动。63,64]自恋人格反过来又被发现与成瘾有关。65]这一发现将在关于成瘾的一节中进行更详细的讨论。
此外,似乎不同人格特征的人对SNSS的使用也不尽相同。66]并且更愿意使用脸书[33]性格外向、经验开放的人更常使用SNSS,前者适用于成熟,后者适用于年轻人。66]此外,外向的人和对经验开放的人是更多群体的成员。Facebook更多地使用社交功能33),并拥有更多脸书朋友比内向的人[67,概括地说前者具有较高的社交能力。68]相反,内向的人会在自己的网页上透露更多的个人信息。67]此外,似乎特别害羞的人会花大量的时间在脸书在这个SNS上有很多朋友69]因此,对于那些现实生活网络受到限制的人来说,SNSS似乎是有益的,因为他们可以很容易地接触到同龄人,而不需要现实生活中的亲近和亲密。这种方便性要求这一群体承担更多的时间承诺,这可能导致过度使用和/或可能使人上瘾的使用。
同样,有神经质的男性比有神经质的女性更频繁地使用SNS。66]此外,神经症患者(一般)倾向于使用Facebook的墙功能,他们可以在那里接受和发布评论,而神经质得分低的人更喜欢贴照片。33]这可能是因为神经质的个人对基于文本的帖子而不是视觉显示的情绪内容有更大的控制。33]然而,另一项研究[67相反,那些在神经质上得分较高的人更倾向于把他们的照片贴在自己的页面上。一般来说,神经质的发现意味着那些在这个特质上得分较高的人披露信息是因为他们在网上寻求自我保证,而那些得分较低的人在情感上是安全的,因此为了表达自己而分享信息。67]而较高的自我披露率则与主观幸福感的测量值呈正相关[57]这是否意味着对SNSS的低自我披露可能与潜在成瘾的风险更高有关,这仍然值得怀疑。通过在页面上披露更多的个人信息,用户会面临负面反馈的风险,而负面反馈与幸福感下降有关。70]因此,在未来的研究中,关于SNS的自我披露与成瘾之间的关系需要进行实证研究。
在亲和性方面,研究发现,在这一特征上得分高的女性上传的图片显著多于得分较低的女性,而男性则相反。67]除此之外,那些高度认真的人被发现有更多的朋友,并且上传的照片比那些在这个人格特征上得分低的人要少得多。67]这一发现的一个解释可能是,认真负责的人更倾向于培养他们的在线和线下联系人,而不必公开分享过多的个人信息。
总的来说,这些研究的结果表明,外向的人使用SNS来增强社会,而内向的人则用它来进行社会补偿,每一种社会补偿似乎都与更多的SNS使用有关。关于成瘾问题,由于不同的原因,这两个群体都有可能产生上瘾倾向,即社会增强和社会补偿。此外,在未来的研究中,关于内向的朋友数量的不同研究结果值得更仔细的研究。神经质方面的结果也是如此。一方面,神经症患者经常使用SNS。另一方面,研究表明,对神经质得分高的人的使用偏好不同,这需要进一步的研究。此外,这些互联网应用的结构特点,(i.e.,他们以自我为中心的建设)似乎允许有利的自我披露,这吸引自恋者使用它。最后,亲和性和责任心似乎与SNS的使用程度有关。与自恋、神经质、外向性和内向性人格特征相关的高使用率可能意味着这些群体中的每一个人都特别容易产生使用SNS成瘾的风险。
3.4.负相关
一些研究强调了大量使用SNS的潜在负面关系。例如,一项对184名互联网用户的在线调查结果表明,使用SNS的人在使用上花费的时间较多,人们认为他们很少参与实际生活中的社区活动。71]这类似于这样一种发现,即那些对自己与同龄人的现实生活联系感到不安全、因而具有负面社会身份的人,倾向于更多地使用snss来弥补这一问题。37]此外,SNS的使用似乎决定了SNS的使用对幸福感和自尊的影响。
更具体地说,10至19岁的荷兰青少年,如果得到的是主要的负面反馈,那么他们的自尊就会降低,进而导致他们的幸福感下降。70]考虑到人们上网时往往不受拘束72在互联网上,给予和接受负面反馈可能比在现实生活中更为普遍。这可能会带来负面后果,特别是对那些自尊心不高的人来说,他们倾向于用sns来弥补现实社会网络的不足,因为他们依赖于通过这些网站得到的反馈。43]因此,潜在的是,自尊心较低的人有可能养成使用SNSS成瘾的风险。
根据最近的一项评估脸书抽样调查219名大学生的使用情况及学业成绩[73], 脸书与没有使用这个SNS的学生相比,用户的平均分数更低,学习时间也更少。在26%的学生报告他们的使用对他们的生活有影响,四分之三(74%)声称它有一个负面的影响,即拖延,分心,和糟糕的时间管理。对此,一个潜在的解释可能是,使用互联网学习的学生可能因同时参与学习网络而分心,这意味着这种形式的多任务处理不利于学业成绩。73].
此外,似乎脸书在某些情况下可能会对浪漫关系产生负面影响。在某人身上披露丰富的私人信息脸书包括状态更新、评论、图片和新朋友的页面,可能会导致嫉妒的网络跟踪[74,包括人际电子监察;[75)由某人的伴侣。据报道,这会引起嫉妒[76,77在最极端的情况下,离婚和相关的法律诉讼[78].
这几项现有的研究强调,在某些情况下,SNS的使用会导致各种负面后果,这意味着潜在的参与现实社会的减少,更糟糕的学习成绩,以及关系问题。减少和危害学术、社会和娱乐活动被视为物质依赖的标准[18因此可被视为行为成瘾的有效标准[79],如SNS成瘾。有鉴于此,赞同这些标准似乎会使人们有发展成瘾的风险,前面各段概述的科学研究基础支持SNSS潜在的成瘾质量。
尽管有这些发现,但由于本研究中缺乏纵向设计,无法就过度使用SNSS是否是所报告的负面后果的因果因素作出任何因果推断。此外,需要考虑到潜在的混乱因素。例如,大学生学习中的多任务问题似乎是与学业成绩差有关的一个重要因素。此外,在浪漫伴侣的情况下,预先存在的关系困难可能会因使用社交网络而加剧,而后者并不一定是随后问题背后的主要驱动力。然而,研究结果支持了这样一种观点,即有些人使用SNS来应对负面的生活事件。相应地,应对被发现与物质依赖和行为成瘾都有关联。80]因此,声称不正常的应对措施之间存在联系的说法似乎是正确的。i.e.逃避和逃避)和过量使用SNS/成瘾。为了证实这一猜想,并更充分地研究与SNS使用相关的潜在负相关关系,还需要进一步的研究。
3.5.毒瘾
研究人员认为,过度使用新技术(特别是在线社交网络)可能特别容易使年轻人上瘾。81]按照生物心理社会框架研究成瘾的病因[16]和成瘾综合症模型[17],据称,那些沉迷于使用SNSS的人所经历的症状与那些对物质或其他行为上瘾的人所经历的症状相似。81]这对临床实践有重大影响,因为与其他成瘾不同,SNS成瘾治疗的目标不能是完全戒断使用互联网。本身因为后者是当今专业和休闲文化不可或缺的组成部分。相反,最终的治疗目标是控制互联网的使用及其各自的功能,特别是社交网络应用,以及使用认知行为疗法中制定的策略来预防复发[81].
除此之外,学者们还假设,具有自恋倾向的易受伤害的年轻人特别容易上瘾。65]迄今为止,只有三项实证研究在同行评议的期刊上进行和发表,这些期刊专门评估了SNSS的成瘾潜力[82–84]除此之外,两篇公开发表的硕士论文分析了SNS成瘾问题,并将于随后提交,目的是为了包容和相对缺乏关于这一主题的数据[85,86]在第一次研究中[83],233名本科生(64%为女生,平均年龄=19岁,SD=2年),采用前瞻性设计方法,通过计划行为理论的扩展模型(Tpb),预测SNSS的高层次使用意图和实际高水平使用。87])。高水平的使用被定义为每天至少使用四次SNSS。TPB变量包括使用意向、态度、主观规范和感知行为控制(PBC)。此外,自我认同(改编自[88)归属[89],并对SNSS过去和将来的应用进行了探讨。最后,使用Likert量表上的8个问题对成瘾倾向进行评估。90]).
在完成第一份问卷一周后,参与者被要求说明在过去一周内,他们每天至少访问四次国家统计局的天数。研究结果表明,过去的行为、主观规范、态度和自我认同对行为意向和实际行为都有显著的预测作用。另外,自我认同和归属感对sns使用的成瘾倾向有显著的预测作用。83]因此,那些自称是SNS用户的人和那些在SNS上寻找归属感的人似乎有可能对SNS上瘾。
在第二项研究中[82],澳大利亚大学生抽样201人(76%为女性,平均年龄=19,SD=2)是为了通过近地天体人格问卷(neo-FFI)的简短版本来评估人格因素;[91),自尊清单(SEI;[92),使用SNSS的时间,以及成瘾倾向量表(基于.90,93])。成瘾倾向量表包括三项测量显着性、失控和戒断的项目。多元回归分析结果表明,高外向性和低自觉性得分显着地预测了成瘾倾向和使用SNS的时间。研究人员认为,外向性与成瘾倾向之间的关系可以通过以下事实来解释:使用SNSS可以满足外向者社交的需要。82]关于缺乏责任心的发现似乎与以往关于一般互联网使用频率的研究相一致,因为那些在责任心上得分较低的人使用互联网的频率要高于那些在这一人格特质上得分较高的人。94].
在第三项研究中,Karaiskos等人 [84]报告一位24岁女性使用SNS的情况,她的行为严重影响了她的职业和私人生活。因此,她被转介到一家精神科诊所。她用脸书一天至少有五个小时被解雇,因为她不停地检查自己的SNS而不是工作。即使在接受临床采访时,她也用手机上网。脸书。除了过度使用会导致女性生活各方面的严重损害外,她还会出现焦虑症状和失眠,这暗示了SNS成瘾的临床相关性。这种极端的情况导致一些研究人员将SNS成瘾概念化为网络频谱成瘾[84]这表明,第一,SNS成瘾可以在更大的网络成瘾框架内分类,第二,它是一种特定的网络成瘾,与其他成瘾的互联网应用一样,如网络游戏成瘾[95互联网赌博成瘾[96和网瘾[97].
在第四项研究中[85],通过网络成瘾测试对SNS游戏成瘾进行评估[98]使用342名年龄在18至22岁的中国大学生。在本研究中,sns游戏成瘾指的是对sns游戏上瘾。快乐农场。学生们被定义为沉迷于使用这个SNS游戏,因为他们认可了IAT八项中至少五项的内容。使用这种切断,24%的样本被确定为成瘾[85].
此外,作者还对SNS游戏使用的满足感、孤独感进行了调查。99休闲无聊[100和自尊[101]结果表明,孤独感与SNS游戏成瘾呈弱正相关,休闲无聊与SNS游戏成瘾呈中度正相关。此外,社交群体中的“包容”、“成就”(游戏)、休闲无聊和男性性别显著预测了SNS游戏成瘾[85].
在第五项研究中[86对335名年龄在19~28岁的中国大学生进行了青少年网络成瘾测试。98]修改为专门评估对普通中文SNS的上瘾,即Xiaonei.com。当用户在IAT中指定的8项成瘾项目中,有5项或更多项目时,他们被归类为成瘾。此外,作者还评估了孤独99]、用户满意程度(基于先前焦点小组访谈的结果)、SNS网站的使用属性和模式[86].
结果表明,在全部样本中,34%被归类为成瘾者。此外,孤独感与使用频率和会话时间呈显著正相关。Xiaonei.com还有SNS成瘾。同样,社交活动和人际关系的建立也可以预测社交网络成瘾[86].
不幸的是,从批评的角度来看,这里审查的定量研究受到各种限制。最初,仅仅评估成瘾倾向并不足以界定真正的病理。此外,这些样本在女性性别方面是小的、具体的和偏斜的。这可能导致很高的成瘾流行率(高达34%)报告[86]显然,需要确保的是,与其评估过度使用和/或过度职业,还需要具体评估吸毒成瘾情况。
威尔逊等人研究[82只认可三个潜在的成瘾标准,这不足以在临床上确立成瘾状态。同样,严重的损害和消极后果也将成瘾与单纯的虐待区分开来[18在本研究中根本没有进行评估。因此,今后的研究具有很大的潜力,可以通过采用更好的方法设计,包括更有代表性的样本,并使用更可靠和更有效的成瘾量表,以填补目前在经验知识方面的空白,从而解决因特网上使用社交网络成瘾的现象。
此外,研究必须解决除消极后果之外的特定成瘾症状。这些标准可参照DSM-IV tr物质依赖标准[18以及依赖综合征的ICD-10标准[102],包括:(1)容忍;(2)退出;(3)增加使用;(4)失去控制;(5)延长恢复期;(6)牺牲社会、职业和娱乐活动;(7)继续使用,尽管产生不利后果。这些已被发现是诊断行为成瘾的适当标准。79],因此似乎足以适用于SNS成瘾。为了被诊断为sns成瘾,上述标准中至少有三项(但最好更多)应在相同的12个月内达到,并且必须对个人造成重大损害[18].
根据这一定性案例研究,从临床角度看,SNS成瘾是一个可能需要专业治疗的心理健康问题。与定量研究不同的是,案例研究强调了个体所经历的重大的个体损伤,这些损伤跨越了不同的生活领域,包括他们的职业生活以及他们的心身状况。因此,建议未来的研究人员不仅要以定量的方式调查SNS成瘾,而且要通过分析SNS过度使用的个体案例来进一步了解这一新的心理健康问题。
3.6.特异性与共病性
似乎必须充分注意(一)SNS成瘾的特殊性和(二)潜在的共病。霍尔等人 [103]概述为什么有必要解决精神障碍(如成瘾)之间的共病问题的三个原因。首先,大量精神疾病的特点是额外的(亚)临床问题/障碍。第二,必须在临床实践中解决共同条件,以提高治疗效果。第三,可以制定具体的预防方案,其中包括特别针对相关心理健康问题的不同层面和治疗方式。由此可以看出,评价SNS成瘾的特异性和潜在的共存性是非常重要的。然而,到目前为止,解决这一问题的研究几乎是不存在的.几乎没有关于SNS成瘾与其他类型成瘾行为共存的研究,主要是因为对SNS成瘾的研究很少,如前一节所强调的那样。然而,在较小的实证基础上,可以对与社交网络成瘾相关的共同成瘾现象做出一些推测性的假设。
首先,对于一些人来说,他们的社交网络成瘾占用了大量的可用时间,除非其他行为成瘾可以通过社交网站(如赌博成瘾、游戏成瘾)找到出路,否则很难与其他行为成瘾同时发生。简单地说,同一个人几乎没有面子上的有效性,例如,工作狂和社交网络成瘾者,或者运动成瘾者和社交网络成瘾者,主要是因为每天有大量的时间同时进行两种行为成瘾是非常不可能的。尽管如此,有必要确定各自的成瘾行为,因为其中一些行为实际上可能会同时发生。在其中一项研究中,Malat和他的同事们在一项临床样本中诊断出了物质依赖。104]发现61%的人至少有1%和31%的人从事两种或更多的有问题的行为,如暴饮暴食、不健康的人际关系和过度使用互联网。因此,虽然对工作和使用SNS等行为的同时成瘾是相对不可能的,但SNS成瘾可能与暴饮暴食和其他过度久坐行为同时发生。
因此,第二,从理论上讲,社交网络成瘾者有可能有额外的毒瘾,因为同时进行行为成瘾和化学成瘾是完全可行的。16]从动机的角度来看,这也可能是有意义的。例如,如果社交网络成瘾者从事这一行为的主要原因之一是因为他们的自尊心很低,那么有些化学成瘾可能会达到同样的目的,这是很有直觉的。因此,研究表明,成瘾行为在患有物质依赖的人中是相对常见的。在一项研究中,布莱克等人 [105]发现样本中38%的有问题的计算机用户除了行为问题/成瘾外,还患有药物使用障碍。显然,研究表明,一些网瘾患者同时也有其他上瘾的经历。
在1,826名接受药物成瘾(主要是大麻成瘾)治疗的病人样本中,发现有4.1%的人患有网瘾[106]此外,进一步研究的结果[107]指出青少年网瘾和吸毒经验有共同的家庭因素,即父母-青少年冲突较多、兄弟姐妹习惯性饮酒、父母对青少年药物使用的积极态度、家庭功能低下等。此外,林等人 [108]对13至18岁的1 392名青少年进行了互联网成瘾及其相关因素的评估。在潜在的共病方面,他们发现通过网络成瘾测试,饮酒行为是被诊断为网瘾的一个危险因素。109]这意味着酗酒/依赖可能与SNS成瘾有关。对此的支持来自Kuntsche。等人 [110]他们发现,在瑞士青少年中,社会认可的期望与酗酒问题有关。由于SNSS本质上是人们为社交目的而使用的社交平台,因此可以合理地推断,确实有一些人患有共同成瘾,即SNS成瘾和酗酒。
第三,SNS成瘾的特异性可能与人格特征有关。高等人 [111]发现青少年的高新颖性寻求(NS)、高伤害回避(HA)和低回报依赖(RD)可预测网络成瘾(IA)。上网成瘾、有吸毒经验的青少年在NS上得分显著高于IA组,HA评分低于IA组。因此,医管局似乎特别影响网瘾的专一性,因为高医管局会将互联网成瘾者与那些不仅对互联网上瘾,而且使用药物的人加以区分。因此,似乎有可能假设避免伤害程度较低的人有可能产生对SNS和物质的共同上瘾。因此,研究需要专门针对那些沉迷于使用SNS的人来解决这一差异,以便将这种潜在的紊乱与共患条件区分开来。
此外,似乎有理由具体讨论人们可以在其SNS上开展的活动。已经有一些研究人员开始研究社交网络和赌博之间的可能关系。112–116以及社交网络和游戏[113,116,117]所有这些著作都指出了社交网络媒体是如何用于赌博和/或赌博的。例如,在线扑克应用程序和社交网站上的在线扑克团体是最受欢迎的[115),其他人也注意到媒体对社交网络游戏上瘾的报道,例如范维尔 [117]虽然到目前为止还没有实证研究通过社交网络对赌博或电玩上瘾,但没有理由怀疑那些在社交网络媒体上玩的人比那些玩其他在线或线下媒体的人更容易沉迷于赌博和/或电玩。
总之,解决SNS成瘾和其他成瘾的特殊性是必要的,因为:(1)将这种疾病理解为不同的心理健康问题,同时(2)尊重相关条件,这将(3)有助于治疗和(4)预防工作。从所报道的研究来看,个人的成长和心理社会环境似乎是网络成瘾和物质依赖之间潜在的共患病的影响因素,这得到了成瘾及其病因的科学模型的支持。16,17]此外,酒精和大麻依赖被概述为潜在的共同发生的问题。尽管如此,除此之外,本研究并没有具体讨论特定物质依赖与个人成瘾行为(如使用SNSS成瘾)之间的离散关系。因此,需要对SNS成瘾的特异性和共病性进行进一步的实证研究。
4.讨论和结论
本文献综述的目的是概述与互联网上社交网络的使用和成瘾有关的新兴的经验性研究。最初,SNS被定义为虚拟社区,为其成员提供利用其固有的Web2.0特性的可能性,即联网和共享媒体内容。瑞士央行的历史可以追溯到20世纪90年代末,这表明它们并不像最初可能出现的那样新。随着SNSS的出现,例如脸书,SNS的总体使用速度加快,以至于被认为是一种全球性的消费现象。今天,超过5亿的用户是活跃的参与者。脸书单是社区和研究表明,55%到82%的青少年定期使用SNSS。从同龄人的SNS页面中提取信息是一项特别令人愉快的活动,它与欲望系统的激活联系在一起,而欲望系统又与成瘾体验有关。
就社会人口统计而言,所提出的研究表明,总体而言,SNS的使用模式不同。女性使用社交网络似乎是为了与同龄人交流,而男性则是为了获得社会补偿、学习和社会认同的满足感[37]此外,男性倾向于在SNS网站上披露更多与女性相关的个人信息[25,118]此外,更多的妇女被发现使用MySpace与男子特别有关[26]此外,性别间的使用模式也因个性的不同而不同。与有神经质特征的女性不同,有神经质特征的男性使用SNS的频率更高。66]除此之外,还发现男性更容易沉迷于SNS游戏,特别是与女性相关的游戏[85]这与一项调查结果一致,即男性一般都有发展网络游戏成瘾的风险。95].
唯一评估使用年龄差异的研究[23]表示后者实际上因年龄而异。具体来说,“银牌冲浪者”(i.e.那些60岁以上的人有一个较小的网络朋友圈,相对于年轻的SNS用户来说,他们的年龄不同。根据目前主要评估青少年和学生样本的经验知识,似乎不清楚老年人是否过度使用SNSS,以及他们是否对使用SNSS上瘾。因此,今后的研究必须以填补这一知识空白为目标。
其次,以需求和满足理论为基础,对使用SNSS的动机进行了回顾。总的来说,研究表明,国家统计局被用于社会目的。总的来说,强调维持与离线网络成员的连接,而不是建立新的联系。在这方面,SNS用户通过与其他SNS用户的各种异构连接来维持连接社会资本的能力。这似乎有利于他们分享与就业和相关领域有关的知识和未来的潜在可能性。实际上,个人通过社交网络获得的知识可以被认为是“集体智慧”[119].
集体智慧扩展了共享知识的概念,因为它不局限于某个特定社区的所有成员共享的知识。相反,它表示每个成员的知识的聚合,这些知识可以被各自社区的其他成员访问。在这方面,谋求在国家安全体系上建立薄弱的联系是非常有益的,因此与满足成员的需要是一致的。同时,这也是一种令人欣慰的体验。因此,与其寻求情感上的支持,个体更多的是利用SNSS来与家人和朋友,以及与更远的熟人保持联系,从而保持与潜在有利环境的脆弱联系。大型在线社交网络的好处可能会导致人们过度使用它们,而这反过来又可能导致上瘾行为。
在人格心理学方面,发现某些人格特征与潜在滥用和/或成瘾相关的高使用率有关。在这些人中,外向和内向是突出的,因为每一个都与更多的习惯参与互联网上的社交网络有关。然而,外向者和内向者的动机不同,因为外向者增强了他们的社交网络,而内向者则弥补了现实社会网络的不足。据推测,更高程度地使用SNS的动机可能与外向者分享的动机有关,这表明需要与他们的社区保持联系和社交。然而,在这些人中,高外向性被发现与潜在的使用社交网络成瘾有关,这是因为他们的责任心不高[82].
对于在各自人格特质上得分较高的成员而言,不同的使用动机可能会为未来对SNSS潜在成瘾的研究提供参考。假设情况下,那些弥补与现实生活中缺乏联系的人,可能会面临更大的成瘾风险。实际上,在一项研究中,通过在这个社区寻找归属感,可以预测SNS成瘾的使用情况[83支持这一猜想。想必,在神经质和自恋方面得分较高的人也是如此,假设这两个群体的成员都有较低的自尊心。研究表明,人们过度使用互联网是为了应对日常压力。120,121]这可能是一个初步的解释,调查结果的负面相关,被发现与更频繁的使用SNS。
总体而言,在社交网络上参与一些特定的活动,如社交搜索,以及被发现与更大程度上使用社交网络相关的个性特征,可能会成为未来研究的基础,从而确定那些有可能在互联网上使用社交网络成瘾的人群。此外,建议研究人员评估与SNS成瘾相关的因素,包括SNS使用的语用、吸引力、沟通和期望,因为这些因素可以根据SNS成瘾的特异性病因框架预测SNS成瘾的病因。15]由于对SNS成瘾特异性和共患病的研究较少,因此有必要进行进一步的实证研究。此外,研究者被鼓励密切关注内向者和外向者的不同动机,因为每种动机似乎都与更高的使用频率有关。更重要的是,研究潜在成瘾与自恋的关系似乎是一个富有成效的实证研究领域。除此之外,还需要解决使用SNS的动机以及与过度使用SNS相关的各种负面因素。
除了上述意义和对未来研究的建议外,还需要特别注意选择更大的样本来代表更广泛的人群,以提高各自研究的外部有效性。结果的可概括性是必要的,以界定有可能发展对SNS上瘾的人群。同样,似乎有必要进行进一步的心理生理学研究,以便从生物学角度评估这一现象。此外,需要评估明确和有效的成瘾标准.对成瘾的研究仅限于评估几个标准是不够的。将病理学与高频率和有问题的使用区分开来,就必须采用国际分类手册建立的框架[18,102]此外,根据临床证据和实践,似乎有必要注意SNS成瘾者由于滥用和/或成瘾行为而在各种生活领域中所经历的重大损害。
同样,基于自我报告的数据也不足以诊断,因为研究表明它们可能是不准确的。122]可以想象的是,自我报告可以通过结构化的临床访谈来补充[123]和进一步的案例研究证据以及用户重要其他方面的补充报告。总之,互联网上的社交网络是五彩缤纷的Web2.0现象,它们提供了成为集体智慧的一部分和利用的潜力。然而,过度使用和成瘾的潜在心理健康后果还有待用最严格的科学方法来探讨。
   
























%Extraversion, neuroticism, attachment style and fear of missing out as predictors of social media use and addiction
%Social media use may lead to social media addiction, which involves being unable to control one's social media use and using it to such an extent that it interferes with other life tasks (Ryan, Chester, Reece, and Xenos, 2014). This paper will examine predictors of social media use and addiction focusing on the personality traits of extraversion and neuroticism, attachment style, and the fear of missing out (FOMO).
%Extraversion has been shown to be positively related to both social media use and addictive tendencies (Kuss and Griffiths, 2011, Ryan and Xenos, 2011, Wilson et al., 2010). Extraverts appear to use social media in order to enhance their social connections (Kuss and Griffiths, 2011). Neuroticism has also been shown to be positively associated with social media use (Tang, Chen, Yang, Chung, and Lee, 2016) and internet addiction (Andreassen et al., 2013, Tsai et al., 2009). People high in neuroticism may be drawn to use social networking sites like Facebook because they hope to receive feedback and reassurance from others and because it is easier for them to communicate through a screen than it is for them to communicate face-to-face (Kandell, 1998).
%Because social media is generally used to maintain and develop relationships, attachment style may affect its use. Anxiously attached people are insecure in relationships and often seek reassurance. They may use social media to maintain relationships and seek social feedback. Furthermore, communication through social media can help those who are anxious spend more time thinking about what they want to say and avoid awkward pauses that may occur in real conversations (Kandell, 1998). Research has found that anxious attachment is related to using and seeking feedback on social media (Hart et al., 2015, Oldmeadow et al., 2013). The relationship between anxious attachment and addiction is less clear. While some research has found that insecure attachment is related to problematic internet use and internet addiction (Lin et al., 2011, Schimmenti et al., 2014), another study found no differences in social media addiction between attachment styles (Baek, Cho, and Kim, 2014).
%Those high in attachment avoidance consider themselves self-sufficient and avoid intimacy and closeness. One might assume that they may not wish to use social media as they may not be interested in developing and maintaining relationships. Nevertheless, social media may be used by those with avoidant attachment as a way to keep people in their lives, but at a distance (Nitzburg and Farber, 2013). In some research, avoidant attachment has been linked with less social media use (Hart et al., 2015). However, other research found that those who were both anxious and avoidant used social media more than those who were solely avoidant (Baek et al., 2014).
%When people are anxious about relationships, they likely fear being socially excluded. Fear of missing out (FOMO) is a fear that other people are having fun without you (Przybylski, Murayama, DeHaan and  Gladwell, 2013). FOMO has been linked to increased social media use (Przybylski et al., 2013), as well as to problematic smartphone use (Elhai, Levine, Dvorak and Hall, 2016). However, to our knowledge, no research has specifically examined the relationship between FOMO and social media addiction.
%In our study, we examined the effects of extraversion, neuroticism, attachment styles, and levels of FOMO on both social media use and addiction. We used hierarchical regression, first entering extraversion and neuroticism, then anxious and avoidant attachment, and lastly, FOMO. We hypothesized that each variable would be a significant predictor of both use and addiction at each step.
%1. Method
%1.1. Participants
%We recruited 207 participants (50 men, 155 women, and two who indicated that their gender was “other”). The majority were recruited from a general psychology subject pool in the Southeastern U.S. (n = 118, 57%) and 89 (43%) were recruited online through Facebook or Reddit. Participants ranged in age from 17 to 49, (M = 22.15, SD = 7.38). Most identified as White/Caucasian (79.2%) and 82% were currently enrolled in college.
%1.2. Procedure
%We posted the link to the survey on Facebook and Reddit. The survey was also made available to general psychology students in order to receive partial credit for their course. We merged the data for both of the groups and differences between the groups were tested before our primary data analysis.
%1.3. Materials
%1.3.1. Fear of Missing Out Scale
%This scale (Przybylski et al., 2013) consisted of 10 items measured on a 5-point scale (1 = not at all true to 5 = extremely true). The Cronbach's alpha was 0.91.
%1.3.2. Revised version of the Experience in Close Relationship Scale
%The Experience in Close Relationships Scale (Brennan, Clark and Shaver, 1998) was developed to assess attachment anxiety and avoidance. We utilized a revised version that has been used in research about social media use (Baek et al., 2014). It consists of 5 items measuring anxiety and 5 items measuring avoidance on a 7-point scale (1 = strongly disagree to 7 = strongly agree). The Cronbach's alphas were 0.81 for anxiety and 0.79 for avoidance.
%1.3.3. The Big Five Inventory
%We used the Big Five Inventory (John and Srivastava, 1999) to measure extraversion (8-items) and neuroticism (8-items). It was rated on a 5-point scale (1 = strongly disagree, 5 = stronglyagree). The Cronbach's alphas were 0.88 for extraversion and 0.84 for neuroticism.
%1.3.4. Bergen Social Media Addiction Scale
%This 6-item scale included items measuring whether one was troubled when one could not use social media and whether it interfered with one's job or studies (Andreassen, Torsheim, Brunborg and Pallesen, 2012). It was rated on a 5-point scale (1 = very rarely, 5 = very often). The Cronbach's alpha was 0.88.
%1.3.5. Social Media Engagement Scale
%This 5-item scale measured the extent to which an individual uses social media in their daily lives (Przybylski et al., 2013). It was rated on an 8-point Likert scale (1 = not one day last week, 8 = every day last week). The Cronbach's alpha was 0.85.
%2. Results
%Bivariate correlations between all variables can be seen in Table 1. In order to investigate whether the two subsamples differed, we ran a MANOVA with recruitment style (general psychology or online) as the independent variable and all other variables as the dependent variables. The results were significant, F(8,198) = 6.77, p < 0.001. Univariates indicated that the subsamples differed significantly by age F(1,205) = 48.36, p < 0.001 and social media engagement F(1,205) = 4.74, p = 0.031. Those recruited online were older (M = 25.85, SD = 9.28) than those recruited through the subject pool (M = 19.36, SD = 3.59), and were less engaged with social media (5.03 vs. 5.65). With age covaried, the MANCOVA was not significant F (7,198) = 0.86, p = 0.54 meaning age accounted for the differences between the subsamples. Thus, age was included as the first step to control for its effects in our regression analyses.
%Table 1. Inter-correlations among study variables.
%	1	2	3	4	5	6	7	8
%1. Age	–	− 0.005	− 0.136	− 0.004	− 0.253⁎⁎
%− 0.174⁎
%− 0.163⁎
%− 0.220⁎⁎
%2. Extraversion		–	− 0.239⁎⁎
%− 0.370⁎⁎
%− 0.075	0.010	0.062	0.130
%3. Neuroticism			–	0.274⁎⁎
%0.587⁎⁎
%0.481⁎⁎
%0.274⁎⁎
%0.250⁎⁎
%4. Avoidance				–	0.186⁎⁎
%0.154⁎
%0.160⁎
%− 0.078
%5. Anxiety					–	0.643⁎⁎
%0.342⁎⁎
%0.262⁎⁎
%6. FOMO						–	0.560⁎⁎
%0.357⁎⁎
%7. Addiction							–	0.575⁎⁎
%8. Engagement								–
%Note. n = 207.
%⁎
%p < 0.05.
%⁎⁎
%p < 0.01.
%The hierarchical regressions for both social media use and addiction can be seen in Table 2. For social media use, we found that age was significant at the first step such that younger people used social media more. After adding extraversion and neuroticism, age remained a significant predictor and both extraversion and neuroticism were significant. At the third step, the addition of attachment avoidance and anxiety was not significant, only neuroticism and extraversion were significant predictors at this step. The addition of FOMO in the final step was significant. The final model accounted for 17.1% of the variance F(1,200) = 11.13, p < 0.001 in social media use. Age, neuroticism, and FOMO were significant in the final model.
%Table 2. Regression analyses.
%Variable	Regression analyses
%	Social media use	Social media addiction
%	β	p	β	p
%Step 1:				
% Age	− 0.22	0.001	− 0.16	0.02
%	R2 = 0.044; F = 10.4***
%R2 = 0.027; F = 5.62*
%Step 2:				
% Age	− 0.18	0.005	− 0.12	0.07
% Extraversion	0.19	0.004	0.13	0.07
% Neuroticism	0.27	< 0.001	0.29	< 0.001
%	ΔR2 = 0.085; ΔF = 9.94***
%ΔR2 = 0.081; ΔF = 9.19***
%Step 3:				
% Age	− 0.16	0.20	− 0.09	0.20
% Extraversion	0.15	0.03	0.16	0.02
% Neuroticism	0.22	0.01	0.13	0.13
% Avoidance	− 0.11	0.14	0.14	0.05
% Anxiety	0.12	0.14	0.23	0.0
%	ΔR2 = 0.017; ΔF = 2.05	ΔR2 = 0.053; ΔF = 6.37**
%Step 4:				
% Age	− 0.15	0.02	− 0.08	0.19
% Extraversion	0.12	0.09	0.10	0.12
% Neuroticism	0.17	0.04	0.03	0.72
% Avoidance	− 0.12	0.09	0.12	0.07
% Anxiety	− 0.03	0.76	− 0.07	0.44
% FOMO	0.28	0.001	0.56	< 0.001
%	ΔR2 = 0.045; ΔF = 11.13**
%ΔR2 = 0.173; ΔF = 52.01***
%⁎
%p < .05.
%⁎⁎
%p < .01.
%⁎⁎⁎
%p < .001.
%For social media addiction, age was significant at the first step such that younger people were more likely to be addicted. After adding extraversion and neuroticism, a statistically significant increase, age did not remain a significant predictor but neuroticism was significant. The addition of attachment avoidance and anxiety at the third step resulted in a statistically significant increase. Extraversion, avoidance, and anxiety were all significant at this step. The addition of FOMO in the final model was significant. The final model accounted for 31.4% of the variance in social media addiction, F(1,200) = 52.01, p < 0.000, and FOMO was the only significant predictor.
%3. Discussion
%The goal of our study was to examine whether extraversion, neuroticism, attachment style, and FOMO were significant predictors of social media use and addiction. Previous research has shown that extraversion is a predictor of social media use and addiction (Wilson et al., 2010). Extraversion was a significant predictor of use and addiction, but only predicted addiction at the third step. Extraverted individuals may be more likely to use social media because they crave social interaction; too much use may lead to addiction. On the other hand, addiction may be less of a concern for extraverts because they are also comfortable interacting in person.
%We found that neuroticism was a predictor of use, and predicted addiction when only age and personality variables were entered. This is consistent with previous research that has shown that neuroticism was a predictor of social media use (Tang et al., 2016) and internet addiction (Andreassen et al., 2013). People high in neuroticism may have a lot of anxiety about personal relationships and social media can be used to frequently stay in touch with others. On the other hand, once attachment styles were entered into the regression, neuroticism no longer predicted social media addiction. Thus, the effects of neuroticism on social media addiction may be mediated through insecure attachment styles.
%Interestingly, we found that both anxious and avoidant attachment were predictors of social media addiction before FOMO was included in the model. This is inconsistent with some research that found no relationships between attachment styles and social media addiction (Baek et al., 2014), but consistent with other research that found that insecure attachment styles were associated with internet addiction (Lin et al., 2011, Schimmenti et al., 2014). It may be that avoidant attachment is related to social media addiction only when individuals are also high in attachment anxiety. For such people, social media can be a way to feel connected to others but not actually engage in social interaction (Nitzburg and Farber, 2013).
%Fear of missing out is a relatively newly operationalized variable and previous research showed that it had a positive relationship with social media use (Przybylski et al., 2013). FOMO predicted social media use and addiction above and beyond personality traits and attachment style. Furthermore, our study predicted a greater proportion of the variance in social media addiction than has been found in previous research that has focused on either attachment style or personality characteristics (e.g., Andreassen et al., 2013, Schimmenti et al., 2014). Although research has found that FOMO is linked to problematic smartphone use (Elhai et al., 2016), as far as we know, this is the first study specifically looking at FOMO and social media addiction.
%There were a few limitations to our study. One limitation was having two separate recruitment styles. The participant pool was younger and this accounted for the differences between our recruitment styles on all other variables. By controlling for age, we attempted to account for this limitation. Another limitation was that the majority of our participants were white, college-age, females. Because of this, our study may not generalize to other sociodemographic groups. Lastly, our data is limited by the use of self-report measures; the validity of our data is contingent on the accuracy of our participants' reports.
%Future research should continue to look at FOMO as a contributor to social media use and addiction using more extensive and comprehensive measures. FOMO has recently been linked to negative consequences associated with mobile phone use (Oberst, Wegmann, Stodt, Brand, and Chamarro, 2017), and has been linked to distracted driving (Przybylski et al., 2013). It may also be useful to look at predictors of FOMO. For example, while this study investigated attachment in current close relationships, the history of attachment style one had with one's parents may influence the extent to which people fear social exclusion. Furthermore, other aspects of personality may contribute to this dynamic such as narcissism and loneliness as well as other components of the Big Five such as conscientiousness or agreeableness (Ryan and Xenos, 2011

社交媒体的使用可能会导致社交媒体成瘾,这意味着无法控制自己的社交媒体的使用,并将其使用到干扰其他生活任务的程度。莱恩,切斯特,里斯,谢诺斯,2014年)本文将研究社交媒体使用和成瘾的预测因素,重点是人格特质外向和神经质、依恋风格,以及对错过的恐惧(FOMO)。
外向性被证明与社交媒体的使用和成瘾倾向都呈正相关(Kuss和Griffiths,2011年, Ryan和Xenos,2011年, Wilson等人,2010年)外向的人似乎利用社交媒体来加强他们的社会联系(KussandGriffiths,2011年). 神经质也被证明与社交媒体的使用有着积极的联系(唐、陈、杨、钟、李,2016年)和网瘾(Andreassen等人,2013年, Tsai等人,2009年)高神经质的人可能会被吸引使用像facebook这样的社交网站,因为他们希望从别人那里得到反馈和安慰,而且他们通过屏幕交流比面对面交流更容易。Kandell,1998年).
由于社交媒体通常用于维持和发展人际关系,依恋风格可能会影响其使用。忧心忡忡的人在人际关系中缺乏安全感,常常寻求安慰。他们可以使用社交媒体来维持人际关系,寻求社会反馈。此外,通过社交媒体进行交流可以帮助那些焦虑的人花更多的时间思考他们想说什么,并避免在真实对话中出现尴尬的停顿。Kandell,1998年)研究发现,焦虑依恋与在社交媒体上使用和寻求反馈有关。Hart等人,2015年, Old草甸等人,2013年)焦虑依恋与成瘾之间的关系不太清楚。一些研究发现,不安全的依恋与有问题的互联网使用和网瘾有关。Lin等人,2011年, Schimmenti等人,2014年),另一项研究发现,依恋方式在社交媒体成瘾方面没有差异(Baek,Cho,and Kim,2014年).
那些高度依恋回避认为自己自给自足,并避免亲密和亲密。人们可能会认为,他们可能不希望使用社交媒体,因为他们可能对发展和维持关系不感兴趣。然而,社交媒体可能被那些有回避性依恋的人用来作为一种让人们生活在他们的生活中的方式,但却是在一段距离内。Nitzburgand Farber,2013年)在一些研究中,回避性依恋与社交媒体的使用减少有关(Hart等人,2015年)然而,另一项研究发现,那些既焦虑又回避的人使用社交媒体的次数多于那些单纯逃避的人。Baek等人,2014年).
当人们对人际关系感到焦虑时,他们可能害怕被社会排斥。害怕错过(FOMO)是担心别人在没有你的情况下玩得很开心。Przybylski,Murayama,DeHaan,and Gladwell,2013年)FOMO与社交媒体的使用增加有关(Przybylski等人,2013年),以及有问题的智能手机使用(Elhai,Levine,Dvorak,and Hall,2016)然而,据我们所知,没有任何研究专门研究FOMO与社交媒体成瘾之间的关系。
在我们的研究中,我们研究了外向的影响,神经质对社交媒体的使用和成瘾的依恋风格和FOMO的水平。我们采用分层回归,先进入外向性和神经质,然后是焦虑和回避性依恋,最后是FOMO。我们假设每个变量在每一步都是使用和成瘾的重要预测因子。
1. 方法
1.1. 参加者
我们招募了207名参与者(50名男性,155名女性,2名表示他们的性别是“其他”的)。大多数是从美国东南部的一个普通心理学学科中招募的。n = 118人(57%)和89人(43%)是通过Facebook或Reddit在线招聘的。参加者年龄介乎17至49岁,(M = 22.15, SD = 7.38)。大多数被确认为白人/白种人(79.2%)和82%的人目前正在上大学。
1.2. 程序
我们在Facebook和Reddit上发布了这个调查的链接。这项调查也提供给普通心理学学生,以便他们的课程获得部分学分。我们合并了两组的数据,并在初步数据分析之前对各组间的差异进行了测试。
1.3. 材料
1.3.1. 害怕失去规模
这个比例(Przybylski等人,2013年)由10个项目组成,用5分量表(1)来衡量。 = 一点也不对至5 = 极真). The Cronbach's alpha was 0.91.
1.3.2. 亲密关系经验量表修订版
亲密关系体验量表(Brennan,Clark,and Shaver,1998)用于评估依恋、焦虑和回避。我们使用了一个已用于社交媒体使用研究的修订版(Baek等人,2014年)它包括5项测量焦虑的项目和5项以7分量表衡量回避的项目(1项)。 = 强烈反对至7 = 坚决同意)Cronbach‘s alpas的焦虑和回避分别为0.81和0.79。
1.3.3. 五大库存
我们使用了五大库存(约翰和斯里瓦斯塔瓦,1999年)测量外向性(8项-项目)和神经质(8-项目)。它的评级为5分标准(1)。 = 强不同意,5 = 强同意)。Cronbach‘s阿尔法的外向性为0.88,神经质为0.84。
1.3.4. 卑尔根社会媒体成瘾量表
这个6项量表包括测量一个人在不能使用社交媒体时是否有问题,以及它是否干扰了一个人的工作或学习。Andreassen,Torsheim,Brunborg,and Pallesen,2012)它的评级为5分标准(1)。 = 极少数, 5 = 经常). The Cronbach's alpha was 0.88.
1.3.5. 社会媒体参与量表
这个5项量表衡量个人在日常生活中使用社交媒体的程度(Przybylski等人,2013年)它的评级为8分。Likert标度 (1 = 上周一天都没有, 8 = 上周每天). The Cronbach's alpha was 0.85.
2. 结果
所有变量之间的二元相关性可以在表1。为了考察这两个子样本是否有差异,我们以招聘风格(一般心理学或在线)为自变量,以所有其他变量为因变量,运行了一个Manova模型。结果非常显著,F(8,198) = 6.77, p < 0.001。Univariates表明,各子样本的年龄差异很大。F(1,205) = 48.36, p < 0.001与社交媒体参与F(1,205) = 4.74, p = 0.031。在网上招聘的人年龄较大(M = 25.85, SD = 9.28)比通过主题池征聘的人员(M = 19.36, SD = (3.59)与社交媒体接触较少(5.03比5.65)。随着年龄的变化,MANCOVA没有显着性。F (7,198) = 0.86, p = 0.54岁是各子样本间差异的主要原因。因此,在我们的回归分析中,年龄被作为控制其影响的第一步。
%表1. 研究变量之间的相互关系。
%	1	2	3	4	5	6	7	8
%1.年龄	–	− 0.005	− 0.136	− 0.004	− 0.253⁎⁎
%− 0.174⁎
%− 0.163⁎
%− 0.220⁎⁎
%2.外向		–	− 0.239⁎⁎
%− 0.370⁎⁎
%− 0.075	0.010	0.062	0.130
%3.神经质			–	0.274⁎⁎
%0.587⁎⁎
%0.481⁎⁎
%0.274⁎⁎
%0.250⁎⁎
%4.回避				–	0.186⁎⁎
%0.154⁎
%0.160⁎
%− 0.078
%5.焦虑症					–	0.643⁎⁎
%0.342⁎⁎
%0.262⁎⁎
%6.FOMO						–	0.560⁎⁎
%0.357⁎⁎
%7.毒瘾							–	0.575⁎⁎
%8.订婚								–
%注. n = 207.
%⁎
%p < 0.05.
%⁎⁎
%p < 0.01.
%社交媒体的使用和成瘾的等级回归可以从表2。

对于社交媒体的使用,我们发现年龄在第一步就很重要,所以年轻人更多地使用社交媒体。在加入外向和神经质年龄仍然是一个显著的预测因子,外倾和神经质都有显着性。第三步,依恋回避和焦虑的增加不显著,只有神经质和外向是这一阶段的显著预测因子。在最后一步中加入FOMO是非常重要的。最终模型占方差的17.1%F(1,200) = 11.13, p < 0.001在社交媒体上的使用。年龄、神经质和FOMO在最终模型中有显着性意义。
%表2. 回归分析
%变量	回归分析
%	社交媒体使用	社交媒体成瘾
%	β	p	β	p
%步骤1:				
% 年龄	− 0.22	0.001	− 0.16	0.02
%	R2 = 0.044;F = 10.4***
%R2 = 0.027;F = 5.62*
%步骤2:				
% 年龄	− 0.18	0.005	− 0.12	0.07
% 外向	0.19	0.004	0.13	0.07
% 神经质	0.27	< 0.001	0.29	< 0.001
%	ΔR2 = 0.085; ΔF = 9.94***
%ΔR2 = 0.081; ΔF = 9.19***
%步骤3:				
% 年龄	− 0.16	0.20	− 0.09	0.20
% 外向	0.15	0.03	0.16	0.02
% 神经质	0.22	0.01	0.13	0.13
% 回避	− 0.11	0.14	0.14	0.05
% 焦虑症	0.12	0.14	0.23	0.0
%	ΔR2 = 0.017; ΔF = 2.05	ΔR2 = 0.053; ΔF = 6.37**
%步骤4:				
% 年龄	− 0.15	0.02	− 0.08	0.19
% 外向	0.12	0.09	0.10	0.12
% 神经质	0.17	0.04	0.03	0.72
% 回避	− 0.12	0.09	0.12	0.07
% 焦虑症	− 0.03	0.76	− 0.07	0.44
% FOMO	0.28	0.001	0.56	< 0.001
%	ΔR2 = 0.045; ΔF = 11.13**
%ΔR2 = 0.173; ΔF = 52.01***
%⁎
%p < .05.
%⁎⁎
%p < .01.
%⁎
%p < .001.
对于社交媒体成瘾来说,年龄在第一步就很重要,因此年轻人更容易上瘾。在加外倾和神经质后,有统计学意义的增加,年龄并不是一个显著的预测因素,但神经质是显着性的。第三步增加的依恋、回避和焦虑导致了统计学上的显著增加。在这一步中,外向、回避和焦虑都是显著的。在最终模型中加入FOMO是非常重要的。最后一个模型占社交媒体成瘾差异的31.4%,F(1,200) = 52.01, p < 0.000,而FOMO是唯一有意义的预测因子。
3. 讨论
我们的研究目标是检验是否外向,神经质依恋方式和FOMO是社交媒体使用和成瘾的显著预测因子。先前的研究表明,外向性是社交媒体使用和成瘾的预测因子。Wilson等人,2010年)外向性是使用和成瘾的重要预测因子,但仅在第三步预测成瘾。外向的人更有可能使用社交媒体,因为他们渴望社交互动;过多的使用可能导致上瘾。另一方面,对于外向的人来说,上瘾可能不那么令人担心,因为他们也很乐意与人进行面对面的交流。
我们发现神经质是一个使用的预测因素,并预测只有年龄和个性变量进入成瘾。这与先前的研究相一致,该研究表明神经质是社交媒体使用的一个预测因子。Tang等人,2016年)和网瘾(Andreassen等人,2013年)神经质高的人可能对人际关系有很大的焦虑,社交媒体可以经常与他人保持联系。另一方面,一旦依恋方式进入回归,神经质不再预测社交媒体成瘾。因此,神经质对社交媒体成瘾的影响可能是通过不安全的依恋方式来实现的。
有趣的是,在FOMO纳入模型之前,我们发现焦虑和回避依恋都是社交媒体成瘾的预测因子。这与一些发现依恋风格与社交媒体成瘾之间没有关系的研究是不一致的。Baek等人,2014年),但与其他发现不安全的依恋方式与网瘾有关的研究一致(Lin等人,2011年, Schimmenti等人,2014年)回避性依恋可能与社交媒体成瘾有关,只有当个体也有较高的依恋焦虑时。对于这样的人来说,社交媒体可以是一种感觉与他人有联系的方式,而不是真正参与社会互动的一种方式。Nitzburg and Farber,2013年).
害怕错过是一个相对较新的操作变量,先前的研究表明,它与社交媒体的使用有着积极的关系。Przybylski等人,2013年)FOMO预测社交媒体的使用和成瘾人格特质以及依恋风格。此外,我们的研究预测了社交媒体成瘾差异的比例比以往的研究中发现的更大,之前的研究要么关注依恋风格,要么关注个性特征(例如,Andreassen等人,2013年, Schimmenti等人,2014年)尽管研究发现FOMO与有问题的智能手机使用有关(Elhai等人,2016年据我们所知,这是第一项专门研究FOMO和社交媒体成瘾的研究。
我们的研究有一些局限性。一个限制是有两种不同的征聘方式。参与者库更年轻,这就解释了我们在所有其他变量上的招聘风格的差异。通过控制年龄,我们试图解释这一限制。另一个限制是,我们的大多数参与者是白人,大学年龄,女性。正因为如此,我们的研究可能不会推广到其他社会人口群体。最后,我们的数据受到使用自我报告措施的限制;我们的数据的有效性取决于参与者报告的准确性。
今后的研究应继续将FOMO视为社交媒体使用和成瘾的贡献者,采取更广泛和全面的措施。FOMO最近被认为与手机使用相关的负面后果有关(Oberst,Wegmann,Stodt,brand,and Chamarro,2017),并与分心驾驶有关(Przybylski等人,2013年)研究FOMO的预测因素也可能是有用的。例如,虽然这项研究调查了当前亲密关系中的依恋,但与父母的依恋方式的历史可能会影响人们害怕社会排斥的程度。此外,人格的其他方面也可能促成这一动态,如自恋和孤独,以及五大的其他组成部分,如责任心或随和性(Ryan and Xenos,2011年)最后,可以测试一种中介模型,神经质可以预测依恋焦虑,而依恋焦虑又可以预测社交媒体成瘾。


