\color{violet}
%\chapter{ 电子游戏对儿童青少年认知控制的影响}

 
\section{电子游戏定义、分类、人机强交互环境下的电子游戏有什么新特征  
(韩搏康)
}

一、电子游戏成瘾的定义与部分发展历程 

 %来自维基百科,成瘾定论的争议和历史
亚洲一些国家和研究中心对年轻男性的研究发现,当人全神贯注于互联网游戏时,他们大脑的活动与毒品成瘾者相似,在极端的情况下,可能表现为上瘾行为。但当时,美国精神病学学会将“互联网游戏障碍”归类为“尚待进一步研究”。解放军总医院网瘾治疗中心主任陶然一直坚持“网瘾是一种心理疾病”的观点。他为“网络游戏成瘾”制定的9条诊断标准被收录到美国精神病学会(American Psychiatric Association)发布的DSM-5(《精神疾病诊断与统计手册》,第五版)中。
%(Ref: Characterizing cognitive control abilities in children with 16p11.2 deletion using adaptive ‘video game’ technology: a pilot study  [2018 nature]  \cite{anguera2016characterizing})

2013年,美国精神病学学会发布的第五版《精神疾病诊断与统计手册》(DSM-5)中,引入了“互联网游戏障碍”(IGD,Internet gaming disorder)这一概念,首次明确定义了互联网游戏障碍的症状标准,在12月个内行为满足以下九个诊断标准中的五个或更多个:
\begin{enumerate}
\item 网络对游戏的关注成为日常生活中的主要活动; 
\item 移除互联网时的戒断症状(例如烦躁,焦虑或悲伤,没有药物戒断的体征); 
 \item 花费需要多越来越时间的进行网络游戏(容忍度); 
\item 缺乏对网络游戏的控制;
\item 对以前的爱好和娱乐失去兴趣; 
\item 尽管有心理社会问题的知识,仍继续过度使用; 
\item 欺骗家庭成员,师治疗或其他人关于互联网游戏的数量; 
\item 逃避或缓解消极情绪(例如感到无助,内疚,焦虑); 
\item 丧失生命的重要方面(例如重要的关系,工作或教育/职业机会)
\end{enumerate}

%游戏成瘾的定性在医学界并未获得共识,比如美国精神医学学会认为现有证据不足以证明游戏成瘾是一种精神疾病,并拒绝在第5版(2013年5月)《精神疾病诊断与统计手册》中收录游戏成瘾。而

联合国世界卫生组织(WHO)%则持不同态度,并
在2018年6月将“游戏障碍”收录至第十一版国际疾病分类(缩写:ICD-11),
% “The pattern of gaming behaviour may be continuous or episodic and recurrent. The gaming behaviour and other features are normally evident over a period of at least 12 months in order for a diagnosis to be assigned, although the required duration may be shortened if all diagnostic requirements are met and symptoms are severe.”
给出“电子游戏失调症”的三大判断标准:
\begin{enumerate}
\item 失去对于玩游戏的控制(玩游戏的频率、强度、持续、终止、背景等都要纳入考虑);
 %    “impaired control over gaming (e.g., onset, frequency, intensity, duration, termination, context)”

\item 游戏的重要性高于生活中其他兴趣;
    % .“increasing priority given to gaming to the extent that gaming takes precedence over other life interests and daily activities”

\item 在知悉游戏会产生负面影响的情况下,仍选择继续玩游戏;
 %    “continuation or escalation of gaming despite the occurrence of negative consequences.”
 \end{enumerate}
 世界卫生组织指出,电子游戏成瘾症相关行为要持续至少十二个月才能确诊;若症状严重,确诊前的观察期可缩短。截止??,全球游戏人口中约有2\%到3\%的人存有游戏障碍。
 
二、分类简述:
  电子游戏可以根据内容和体裁分为动作类游戏、冒险类游戏、模仿类游戏、策略类游戏、益智类游戏以及角色扮演类游戏(Schwan, 2006)。动作类游戏主要指对人的反应、手眼协调等要求较高的游戏,例如游戏《反恐精英》、《使命召唤》。冒险类游戏是指玩家以主角的身份参与到一个关于探索和解决问题的故事中来的一种游戏体裁,如游戏《古墓丽影》。模仿类游戏主要是指对现实情境的模仿的游戏,可模拟驾驶、飞行、运动等过程,如游戏《极品飞车》。策略类游戏中,玩家的决策对游戏结果将有重要影响,如游戏《红色警戒》。益智游戏主要强调问题的解决,如《俄罗斯方块》。角色扮演类游戏则主要是玩家扮演某一角色或者控制某一角色,此类游戏和虚幻的场景、故事情节有着较为紧密的联系,如游戏《轩辕剑》。随着游戏的发展,一个游戏也可能同时符合上述分类中的多种游戏标准,例如游戏《机械迷城》是一个冒险类游戏,其中包含《五子棋》等益智类游戏的部分,而且玩家在游戏中主要控制一个角色,因此也可以说是角色扮演类游戏。
                         
         (Ref:      “The Influence of Video Game Training on Cognitive Abilities” May 22 nd , 2014 \cite{})
另:但随着近几年,游戏内容更加丰富,不同种类的游戏之间玩法和内容都有重叠和交叉。单类游戏已经逐渐消失,取而代之的含有多种特点的大型游戏,于是各种游戏的类别又有合并的趋势。(Ref: https://baike.baidu.com/item/游戏类型/360147?fr=aladdin \cite{})

Ref: Paulus, F. W., Ohmann, S., von Gontard, A. and  Popow, C. (2018). Internet gaming disorder in children and adolescents: a systematic review. Developmental Medicine and Child Neurology, 60(7), 645–659. https://doi.org/10.1111/dmcn.13754


\section{对认知控制的影响(双刃剑、好坏两方面) (张仑)
}



%\section{最为关注的社会问题:游戏成瘾   
%\textcolor{Red}{
%(刘鹭语、彭锋)%}
%}

 认知控制(执行控制)是指个体在目标导向的行为(goal-oriented behavior)中,目的性地对其他认知过程(如知觉、注意、工作记忆等)进行自上而下调控的过程。具体而言,认知控制涉及到对信息加工流程的计划、控制和调节。一方面认知控制涉及到对多种基础认知过程的调控,另一方面它又指向与目标完成相关的多种高级认知过程。实验中常用于研究认识控制的范式包括:刺激-反应协同性(与非协同条件相比,在协同条件下需要更大程度的认知控制)、任务转换、错误后反应等。
             %   ------from: 心理所研究揭示不同认知控制过程的时程和频谱特性
  
 %3.addicted online gamers usually aged 13-16(in the Netherland). 
在荷兰, 网络游戏成瘾玩家通常年龄在13到16岁(Ref: Online video game addiction: identification of addicted adolescent gamers a”2010 about cognitive abilities \cite{})

在未成瘾的状态中,电子游戏对认知控制有积极的促进和增强作用。
\begin{enumerate}
\item 空间表征方面: 通过计算机来模拟二维和三维空间的图形,可以改善学生的思维与空间能力。
(F. L. Donelson. The development, testing, and use of a computer interface to evaluate a information processing model describing the rates of encoding and mental rotation in high school students of high and low spatial ability \cite{})
\item 促进了读图能力的提升。(P. M. Greenfield, L. E. Camaioni, and P. Ercolani, et al. Cog-nitive socialization by computer games in two cultures: induc-tive discovery or mastery of an iconic code \cite{})

\item 某些风格的计算机游戏对完成记忆任务大有裨益(还对患有阿尔茨海默氏症的老年人进行精神运动治疗并取得一定成效)
                (  L. Tárrage, M. Boada, and G. Modinos, et al. A randomised pilot study to assess the efficacy of an interactive, multimedia tool of cognitive stimulation in Alzheimer’s disease. Journal of Neurology, Neurosurgery and Psychiatry, 2006, 77(10): 1116-1121. \cite{})
\item  对团队协作能力的正面影响:奥巴马的顾问韦巴赫于 2006 年在他的博客中对”魔兽世界“这种在线多人游戏大加称赞,说它培养了一种合作精神.(势时门户奥巴马顾问玩魔兽世界能培养团队合作网络游戏魔兽世界
http://www.timesk.com/?action-viewnews-itemid-260, 2008-11-29.” \cite{})

\item 对管理能力的影响:网络游戏至少有两个特征能够提升和改善领导力,虚拟游戏经济中的非物质激励,以及高度透明的大量信息,包括有关玩家能力和表现的数据。在“尝试创新”和协调利益”的能力上,玩家的领导力不会比大公司经理差。
(B. Reeves,T. W. Malone, and T. O. Driscoll. Leadership’s onlinelabs [JOL]. http://harvardbusinessonline.hbsp.harvard.edu/hbsp/hbr/articles/article.jsp?ml action=get-article articleID=R0805C, 2008. \cite{})
\end{enumerate}

%部分结论:
%\begin{enumerate}
% %high involvement in playing video games leaves less time for engaging in academic work. 
%\item 电子游戏的投入度越高,从事学术工作的时间就越少。
%%“A brief report on the relationship between self-control, video game addiction and academic achievement in normal and ADHD students” July 22, 2013
% % game addiction decreases significantly GPA and Self-Esteem; it does not influence significantly in self-confidence. 
%\item 游戏成瘾显著降低GPA和自尊,对自信心没有显著影响。
%(Ref:Antecedents and consequences of game addiction \cite{})
%\end{enumerate}




 关于网络游戏成瘾者的认知功能下降的原因可能有三个方面:
\begin{enumerate}
\item 网络游戏成瘾者可能发生了类似于病态赌博和物质成瘾者的脑损害 。(研究证明, 物质致瘾原(海洛因 、鸦片、大麻、酒精等)能 够刺激 大脑内的奖赏神经环路(奖赏中枢)———边缘中脑多巴胺系统产生更多的多巴胺 、乙酰胆碱等神经递质, 使成瘾者产生欣快感。对欣快感的记忆成为继续使用物质致瘾原的动力 ,以致发展为成瘾, 长期的成瘾行为就会引起神经细胞和脑结构发生适应性的退行性改变, 造成对大脑的损害。研究还发现,不仅是物质能够成为致瘾原, 病理性赌博行为也可以刺激神经细胞产生多巴胺 , 病态赌博者在观看赌博画面或者谈论赌博时, 前脑和边缘脑表现出的兴奋和可卡因患者药物渴求时的区域相由于 P300 的发生源包括双侧前额叶、颞叶、顶枕联合区、边缘系统等, 提示网络游戏成瘾存在这些脑区和系统的功能缺陷或障碍, 也表明网络游戏成瘾与其他成瘾涉及到某些相似的脑结构);
\item 网络游戏成瘾使青少年脱离了正常的教育轨道 ,丧失了学习知识 、增长心智能力的机会 ,自然也会阻碍其认知能力的发展;
\item  游戏使他们的“心”逐渐远离现实生活, 而他们又必须“身”在现实之中,无法平衡现实社会(父母、学校、社会)和虚拟世界(网络游戏)之间的矛盾和冲突,容易产生焦虑、孤独、抑郁、强迫等心理问题,甚至精神疾病和自杀,而这些心理异常和精神病性行为又会反过来影响他们的大脑,导致认知功能的降低。
                (Ref:   “网络游戏成瘾者认知功能损害的 ERP研究”2008, \cite{})
 \end{enumerate}





精神疾病诊断和统计手册,第五版(DSM-5)9 在“进一步研究的条件”(第795页)中概述简单描述了“网络游戏障碍”(IGD),表明该提议尚未用于临床使用但鼓励对该主题进行研究. IGD基本的特征是经常参与计算机游戏,通常每天8到10个小时或更长时间,每周至少30个小时,通常在基于互联网的团体游戏中(特别是大型多人在线角色扮演游戏).

%The Diagnostic and Statistical Manual of Mental Disorders, Fifth Edition (DSM‐5) 9 conceptualizes 'Internet gaming disorder' (IGD)  in the chapter 'Conditions for further study' (p.795), suggesting that this proposal is not yet intended for clinical use but that research on this topic is encouraged. The essential feature of IGD is persistent and recurrent participation in computer gaming for typically 8 to 10 hours or more per day and at least 30 hours per week, typically in Internet‐based group games (especially massively multiplayer online role‐playing games [MMORPG]).



DSM-5首次明确定义了IGD的症状标准,在12月个内行为满足以下九个诊断标准中的五个或更多个:(1)网络对游戏的关注成为日常生活中的主要活动; (2)移除互联网时的戒断症状(例如烦躁,焦虑或悲伤,没有药物戒断的体征); (3)花费需要多越来越时间的进行网络游戏(容忍度); (4)缺乏对网络游戏的控制; (5)对以前的爱好和娱乐失去兴趣; (6)尽管有心理社会问题的知识,仍继续过度使用; (7)欺骗家庭成员,师治疗或其他人关于互联网游戏的数量; (8)逃避或缓解消极情绪(例如感到无助,内疚,焦虑); (9)丧失生命的重要方面(例如重要的关系,工作或教育/职业机会)
%For the first time, DSM‐5 has clearly defined symptom criteria for IGD,9 requiring five or more of the following nine diagnostic criteria during a 12‐month period: (1) preoccupation with Internet games becomes the dominant activity in daily life; (2) withdrawal symptoms when removing the Internet (e.g. irritability, anxiety, or sadness, no physical signs of pharmacological withdrawal); (3) need to spend increasing amounts of time with Internet gaming (tolerance); (4) lacking control of Internet gaming; (5) loss of interest in previous hobbies and entertainment; (6) continued excessive use despite knowledge of psychosocial problems; (7) deception of family members, therapists, or others about the amount of Internet gaming; (8) escape from or relief of negative mood (e.g. feeling helpless, guilty, anxious); (9) loss of important aspects of life (e.g. significant relationships, jobs, or educational/career opportunities).
(DMS-5来源于:American Psychiatric Association. Diagnostic and Statistical Manual of Mental Disorders. 5th ed. Arlington, VA: American Psychiatric Publishing, 2013)

3、影响

IGD患者已经描述了缺乏抑制控制,感觉-运动协调和包括自我控制在内的执行控制。 IGD患者描述了执行控制网络的功能连通性较低以及执行功能受损。与健康个体相比,在IGD患者的决策任务中,前岛叶和背外侧前额叶皮层的激活较少。此外,IGD患者的纹状体体积增加,这与认知控制受损有关。尽管存在严重的负面后果,但对冲动性的抑制减少导致残疾控制强迫性游戏。负责情绪控制的杏仁核功能受损可能与情绪增加和即时奖励的增加有关,导致过度游戏而不关注负面的长期后果。

%Deficient inhibitory control, sensory–motor coordination, and executive control including self‐control have been described in patients with IGD. Lower functional connectivity of executive control networks along with impaired executive functions were described in individuals with IGD. Compared with healthy individuals, the anterior insula and the dorsolateral prefrontal cortex are less activated during decision‐making tasks in patients with IGD. In addition, the volume of the striatum is increased in patients with IGD, which relates to impaired cognitive control. Decreased inhibition of impulsivity entails the disability to control compulsive gaming despite serious negative consequences. Impaired functioning of the amygdala, responsible for emotional control, may be associated with increased emphasis of emotions and immediate rewards, leading to excessive gaming without paying attention to negative long‐term consequences.







\subsection{游戏成瘾的定义、表现}

我们将游戏成瘾定义为过度强制使用计算机或电子游戏,导致社交和/或情感问题; 尽管存在这些问题,但游戏玩家无法控制这种过度使用。
%we define game addiction as excessive and compulsive use of computer or videogames that results in social and/or emotional problems; despite these problems, the gamer is unable to control this excessive use.

Ref: Young KS. Internet addiction: the emergence of a new clinical disorder. Cyberpsychol Behav. 1998;1(3):237-44. 


虽然围绕网络成瘾的确切定义仍存在相当大的争议,但对以下症状存在一些共识:a)对互联网的持续关注; b)增加在互联网上花费的时间; c)经常不成功地控制在线时间; d)当减少或中断因特网使用时,用户感到疲倦,不稳或沮丧; e)当用户试图停止使用互联网时易怒; f)与以前计划的相比,互联网上的持久性更长; g)由于使用互联网而危及重要关系甚至专业工作和教育; h)向其他人说谎在互联网上花费的时间;。
%Internet addiction
%While there is still considerable controversy surrounding the exact definition of Internet addiction, there is some consensus on the following symptoms: a) persistent preoccupation with the Internet; b) increasing frequency of the time spent on the Internet; c) frequent unsuccessful attempts to control the time spent online; d) when cut down or interrupted the Internet use, the user feels tired, shaky, or depressed; e) irritability when the user attempts to stop the use of the Internet; f) longer permanence on the Internet in relation to what was previously planned; g) jeopardizing of important relationships or even professional work and education due to the use of the Internet; h) lying to others about the amount of time spent on the Internet; i) use of the Internet as a form of escapism for everyday problems22.


Ref: Khazaal Y, Xirossavidou C. Cognitive-behavioral treatments for internet addiction. Open Addict J. 2012;5(1):30-5.




\subsection{游戏成瘾影响,侧重于认知控制方面的影响}

过度游戏可能会导致一些负面的心理社会后果和心理健康问题,影响可用的时间,工作,教育,家庭,伙伴关系,朋友,社交生活,社会心理健康,社交能力,休闲活动,自尊和孤独。专业和学术问题可能包括成绩差,学业失败和经济问题。学业成绩与病态博弈之间存在负相关关系,影响自尊和自信心。少数游戏玩家报告称游戏对生活质量产生了整体负面影响。此外,Andreassen等人。强调精神疾病如焦虑和抑郁与游戏成瘾之间的关系。 Messias等人。发现每天屏幕时间超过5小时的重度游戏玩家的悲伤,自杀意念和自杀计划风险较高。病理性媒体使用还可以减少睡眠持续时间并破坏睡眠模式。 IGD还与各种躯体健康和医疗后果有关,例如幻听,遗尿,大便,手腕,颈部和肘部疼痛,腱鞘炎('nintendinitis'),肥胖,皮肤水疱,老茧,肌腱疼痛,手臂振动综合征和周围神经病变。此外,IGD的心理社会和医学后果在男性和女性中相似。
%Problematic gaming may lead to several negative psychosocial consequences and mental health problems affecting available time, work, education, family, partnership, friends, social life, psychosocial well‐being, social competence, leisure activities, self‐esteem, and loneliness. Professional and academic problems may include poor grades, academic failure, and financial problems. There is a negative relationship between academic performance and pathological gaming, affecting self‐esteem and self‐confidence. A minority of gamers reported that gaming resulted in an overall negative effect on quality of life. Furthermore, Andreassen et al. emphasized the relationship between psychiatric disorders such as anxiety and depression and gaming addiction. Messias et al. found a higher risk of sadness, suicide ideation, and suicide planning in heavy gamers involved with screen times of more than 5 hours a day. Pathological media use may also reduce sleep duration and disrupt sleep patterns. IGD is also associated with various somatic health and medical consequences, such as auditory hallucinations, enuresis, encopresis, wrist, neck, and elbow pain, tendosynovitis (‘nintendinitis’), obesity, skin blisters, calluses, sore tendons, hand–arm vibration syndrome, and peripheral neuropathy. In addition, psychosocial and medical consequences of IGD are similar in males and females.
%(注:本段来源较多,详见pdf)

Ref: Young, K. (2009). Understanding online gaming addiction and treatment issues for adolescents. American Journal of Family Therapy, 37(5), 355–372. https://doi.org/10.1080/019261
80902942191

由于游戏成瘾者彼此形成一个重要的支持团体并形成亲密关系,因此通常会对婚姻和现实生活关系造成损害。
%As gaming addicts form an important support group with each other and form intimate bonds, the damage is often done to marriages and real life relationships.



%A study on Internet addiction found the most common changes (cognition, behavior and emotion) of these users54. Cognitions: a) flow (the user believes that spent less time than actually spent); b) excessive concern (“If I do not get online, something bad will happen”); c) ruminations (“when I’ll be back online?”); d) denial (“I do not have a problem with the Internet”), and e) unrealistic expectations (“when I go online my life will be much better”). Behaviors: a) avoidance (when confronted with stressful situations, the Internet becomes an escape of everyday problems); b) impulsivity (difficulty in controlling the inappropriate behavior). Emotions: a) craving (urge to use the Internet); b) guilt (when the user realizes the damage of the inappropriate use).

一项关于网络成瘾的研究发现了这些用户最常见的变化(认知,行为和情绪)。认知:a)流程(用户认为花费的时间少于实际花费的时间); b)过度关注(“如果我不上网,会发生不好的事情”); c)反刍(“我何时回到网上?”); d)否认(“我对互联网没有问题”),以及e)不切实际的期望(“当我上网时,我的生活会好得多”)。行为:a)避免(当遇到压力情况时,互联网成为日常问题的逃避者); b)冲动(控制不当行为的困难)。情绪:a)渴望(敦促使用互联网); b)内疚(当用户意识到不当使用的损害时)。


随着成瘾的发展,青少年游戏成瘾者可能会出现戒断症状,包括焦虑,抑郁,烦躁,颤抖的双手,烦躁不安,以及对互联网的痴迷思维或幻想。在线时,他们可能会感到不羁,并且会增加亲密感。当虚拟世界中的关系在重要性上增加时,现实世界中的关系可能被忽略。学业成绩也可能受到影响。
%As the addiction develops, adolescent gaming addicts may experience symptoms of withdrawal, which include anxiety, depression, irritability, trembling hands, restlessness, and obsessive thinking or fantasizing about the Internet. While online they may feel uninhibited and experience an increased sense of intimacy. Relationships in the real world may be neglected as those in the virtual world increase in importance. Academic performance is also likely to suffer.

Ref: Lemmens, J. S., Valkenburg, P. M., and Peter, J. (2009). Development and validation of a game addiction scale for adolescents.Media Psychology, 12(1),77–95.https://doi.org/10.1080/152132
60802669458


游戏玩家会在离线时考虑游戏,并且在他们应该专注于其他事情时经常幻想玩游戏。游戏玩家完全专注于玩游戏,而不是考虑需要完成学业,上课或在图书馆学习的论文。游戏玩家开始错过最后期限,忽视工作或社交活动,因为他们在线并且玩游戏成为他们的主要优先事项。
%Gamers will think about the game when offline and often fantasize about playing the game when they should be concentrating on other things. Instead of thinking about the paper that needs to be completed for school, or going to class, or studying at the library, the gamer becomes completely focused on playing the game. Gamers start to miss deadlines, neglect work or social activities as being online and playing the game becomes their main priority.


对在线游戏的沉迷可能会给游戏玩家带来巨大的后果。游戏上瘾者愿意放弃睡眠,食物和真人接触,只是为了在虚拟世界中体验更多时间。游戏成瘾者有时每天在一次游戏中连续玩十,十五或二十小时。由于游戏的复杂性,玩家在不断变化的虚拟环境中不断受到刺激。随着游戏瘾君子搜索下一次征服或挑战,“再过几分钟”就可以变成几小时。
	%An addiction to online games can cause a tremendous amount of consequences to the gamer. Gaming addicts willingly forgo sleep, food, and real human contact just to experience more time in the virtual world. Gaming addicts sometimes play for ten, fifteen, or twenty hours straight in a single gaming session, every day. Because of the complexity of the game, players are constantly stimulated in an ever-changing virtual environment. “Just a few more minutes” can turn into hours as the gaming addict searches for the next conquest or challenge. 



……刺激使得奖赏回路受到影响同时影响抑制控制回路、动机回路和记忆回路……导致控制回路失能,即使已知……不良后果并告诫自身进行戒除,成瘾者仍然无法控制自己……
	(	\cite{}, 青少年网络游戏成瘾脑结构影像学与认知控制研究)


\section*{成瘾与脑功能的联系}

\begin{enumerate}
\item 背外侧前额叶皮层和前额叶到纹状体回路
\item 双边尾状核功能障碍
\item 多巴胺转运体水平降低……
(主要涉及奖赏系统和控制系统)
\end{enumerate}

\begin{enumerate}
\item 相关性VS因果?
\item 具体神经机制?
\item 被试内实验?
\end{enumerate}

{纵向研究}
\begin{enumerate}
\item 玩游戏前fMRI扫描—>游戏6周—> fMRI扫描

\item 6周药物(抑制多巴胺重摄取)治疗网络成瘾

\item 网络成瘾者不但对奖赏的敏感性增强,而且对损失的敏感性降低 

\end{enumerate}

\section*{更多研究发现}
平行分布式加工(PDP)
背景:
神经传递:相对迟缓、“多噪音”的活动
人的反应速度:视觉刺激到识别和反应-300ms
提出:
以神经活动为根源,涉及人类心智的加工机制  
大脑百亿计的神经元相互联结、神经网络的工作
现状:
以此为基础的认知理论正在构建中……


神经网络/图论分析
弥散张量成像研究发现网络成瘾患者的突显网络异常

网络成瘾患者可能有大规模的脑结构网络改变

图论分析提供量化大脑网络的框架

相关文献较少……

更多探索中
偏好、沉浸、阈限、抑制……


