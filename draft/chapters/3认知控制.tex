\color{magenta}

\section{定义}
认知能力(cognitive ability)(\cite{})是指人脑加工、储存和提取信息的能力,即人们对事物的构成、性能与他物的关系、发展的动力、发展方向以及基本规律的把握能力。它是人们成功的完成活动最重要的心理条件。知觉、记忆、注意、思维和想象的能力都被认为是认知能力。
               % ------from: MBA智库百科(http://wiki.mbalib.com/)
                
 认知控制 (cognitive control)(\cite{}), 也称作执行控制 (executive control) , 是个体在进行目标导向的行为(goal-oriented behavior)时所表现出的高级认知机能,目的性地对其他认知过程(进行自上而下调控的过程。具体而言,认知控制涉及到根据当前任务目标对信息加工流程的计划、控制和调节。一方面认知控制涉及到对多种基础认知过程,如知觉、注意、工作记忆等的调控,另一方面它又指向与目标完成相关的多种高级认知过程。

当需要人们克服自动化的行为, 或者当前任务是不熟悉的、危险的、需要规划和决策的时候, 认知控制能力显得尤为必要 (\cite{Gazzanigaetal2009} 认知控制的模块化组织,  Gazzaniga, Ivry and Mangun, 2009) 。
	
实验中常用于研究认识控制的范式包括:刺激-反应协同性(与非协同条件相比,在协同条件下需要更大程度的认知控制)、任务转换、错误后反应等。   ( Ref: 心理所研究揭示不同认知控制过程的时程和频谱特性 \cite{})


  
  
  
\section{国内外研究现状}

主流研究方法:
\begin{enumerate}
\item 行为学数据:
	字色不一致(color-word Stroop)实验——网络成瘾者反应时间短,错	误次数多;
	 GoStop 任务 ;
\item 影像学-磁共振成像
	前额叶到纹状体网络回路协调运作……
	(Altered brain activation…/Prefrontal cortical modulation…)
\end{enumerate}


\section{内容}



\section*{Conflict and Cognitive Control, Science \cite{}}

%Cognitive control [HN1] is necessary when we block a habitual behavior and instead execute a less-familiar behavior. Because cognitive control requires an effort, it is not efficient to maintain a high level of control all the time—the nervous system needs to know when cognitive control is necessary.
当我们抑制一个习惯性的行为反而去执行一件不太熟悉的事件时,认知控制是必需的。因为认知控制要求付出努力,所以一直保持高水平控制是有失效率的——于是神经系统要知道什么时候有认知控制的需求。

%wo cortical areas in the frontal part of the brain, the anterior cingulate cortex  (ACC) and the lateral prefrontal cortex (LPFC), are considered essential for recruiting cognitive control. This conclusion is based both on the psychological examination of brain-damaged patients and on the imaging of normal human subjects . Botvinick and colleagues have proposed that the ACC detects conflicts between plans of action, and in response to these conflicts recruits greater cognitive control in the LPFC . This hypothesis is consistent with evidence showing the involvement of the LPFC in the execution of cognitive control, such as selective attention and response inhibition. Activation of the ACC by action-plan conflicts has also been reported . However, as yet there is no direct evidence of a connection between the detection of conflicts in the ACC and the subsequent greater control recruited in the LPFC.
大脑额叶的两个皮层区域,前扣带皮层(ACC)和侧前额叶皮层(LPFC)被认为是激起认知控制的必要条件。 该结论基于脑损伤患者的心理检查和正常人类受试者的成像。 Botvinick及其同事推测ACC检测行动计划之间的冲突,并且为了应对这些冲突,LPFC中激活更高水平的认知控制。 该假设与证据表明LPFC参与认知控制的执行有关,例如选择性注意和反应抑制。 还报告了通过行动计划冲突激活行政协调会。 但是,目前尚无直接证据表明行政协调会的冲突检测与随后在LPFC中招募的更大控制措施之间存在联系。

%Cognitive control recruited by the ACC may be “consequential,” that is, based on conflicts between evoked plans of concrete actions. In contrast, in the LPFC, control may be “preemptive,” that is, capable of preventing future conflicts, and may occur at a more strategic level, for example, by increasing attention to the task-related aspects of sensory stimuli. Because neurons selective for different actions are interspersed in local regions, the limited spatial resolution of fMRI might have obscured these action-specific activities in the ACC in previous fMRI studies. Future studies in monkeys and humans should elucidate further the mechanisms defining consequential and preemptive cognitive control and the parts played by the ACC and LPFC.
由ACC激起的认知控制可能是“后果性的”,即基于诱发的具体行动计划之间的冲突。 相反,在LPFC中,控制可以是“抢先的”,即能够防止未来的冲突,并且可以在更具战略性的水平上发生,例如,通过增加对感觉刺激的任务相关方面的关注。 由于对不同行为有选择性的神经元散布在局部区域,因此fMRI的有限空间分辨率可能在先前的fMRI研究中掩盖了ACC中这些特定于行动的活动。 未来对猴子和人类的研究应该进一步阐明定义后果和先发制人认知控制的机制以及ACC和LPFC所扮演的角色。

\section*{ ScienceDirect/Brain and Cognition \cite{}}
%What has fMRI told us about the Development of Cognitive Control through Adolescence?
%Cognitive control, the ability to voluntarily guide our behavior, continues to improve throughout adolescence. 
认知控制,即自发控制行为的能力,在青少年时期持续发展。

%……developmental improvements in inhibitory control may be supported primarily by the ability to establish an inhibitory response state and the brain systems supporting this particular ability.This rationale suggests that what characterizes development through adolescence may not be the emergence of a new cognitive ability (inhibitory control) but the ability to use this tool in a flexible and consistent fashion by effectively establishing a response state. This possibility implies that the circuitry that uniquely supports response state (Dosenbach et al., 2006) is immature in adolescence. Consistent with this possibility, neuroimaging findings indicate that the circuitry supporting response state shows a protracted development through adolescence
抑制性控制的发育改善主要可以通过建立抑制性反应状态的能力和支持这种特定能力的大脑系统来支持。这个理论基础表明,通过青春期发展的特征可能不是新认知能力(抑制控制)的出现,而是通过有效建立响应状态以灵活和一致的方式使用该工具的能力。这种可能性意味着唯一支持响应状态的电路(\cite{}, Dosenbach等,2006)在青春期尚不成熟。 与这种可能性一致,神经影像学发现表明支持反应状态的电路显示出通过青春期的长期发展。

%What is evident across studies is that prefrontal systems and the ability to recruit distributed function are present early in development. However, recent work indicates that the connections within these distributed circuitries increase in strength, and incorporate more long range connections, through adolescence. The transition from adolescence to adulthood therefore can be seen as a change in mode of operation from initially relying on more regionalized processing, such as in the PFC, earlier in development to relying on a broader network of regions that share processing in an efficient and flexible manner at the systems level. Children may rely on processes that support aspects of executive control more generally, while adolescents may transition to utilizing multiple, posterior regions specialized for specific aspects of a task that together provide a rapid response tailored to the task, freeing up executive regions for more complex duties. This transition in the mode of operation may be supported by structural brain changes or other brain dynamics such as brain synchrony that encourage recruitment of more wide-range networks. In adulthood, brain systems may be better specialized and may more efficiently interact with other distant regions, providing a circuitry that supports flexible cognitive control. 
%各研究的明显之处在于,
前额叶系统和激起分布式功能的能力在发育早期就已存在。然而,最近的工作表明,在青春期中,这些分布式电路的连接增加了强度,并建立了更多的长距离连接。因此,从青春期到成年期的过渡可以看作是操作模式的变化,从最初依赖于局部化的处理(例如在PFC中),在发育早期依赖于更广泛区域的网络,这些区域从系统层面以高效灵活的方式共同加工。儿童可能更多地依赖于支持执行控制的过程,而青少年可能会过渡到利用多个针对任务特定方面的后部区域(?),这些区域共同提供任务的快速响应,从而释放执行区域以履行更复杂的职责。这种操作模式的转变可能受到大脑的结构性变化或其他大脑功能的支持,例如能够激活更广范围网络的“大脑同步(brain synchrony)”。在成年期,大脑系统可以更好地专门化,并且可以更高效地与其他远距离区域交互,从而提供支持认知控制的灵活电路。


