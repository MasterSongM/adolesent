% !TEX TS-program = XeLaTeX
\documentclass[12pt]{book}
\usepackage[dvipsnames]{xcolor}
\usepackage{mdframed}
\usepackage{titletoc}
%\usepackage{etoolbox}
\usepackage[colorlinks=blue]{hyperref}


\usepackage{CJK}
\usepackage{geometry}
\geometry{letterpaper}
\usepackage{graphicx,subfigure}

\usepackage{indentfirst}
\setlength{\parindent}{2em}

\usepackage{fontspec, xltxtra, xunicode}
\defaultfontfeatures{Mapping=tex-text}
\setromanfont{SimSun} %设置中文字体
\XeTeXlinebreaklocale “zh”
\XeTeXlinebreakskip = 0pt plus 1pt minus 0.1pt %文章内中文自动换行

%\newfontfamily{\K}{SimHei}
\newfontfamily{\H}{Helvetica} %设定新的字体快捷命令

\usepackage{titlesec} % 改变标题样式
\usepackage{sectsty}


%\chapterfont{\color{blue}}
\sectionfont{\color{blue}}

\graphicspath{{figures/}}


\renewcommand{\contentsname}{目 \quad 录}
%\renewcommand{\abstractname}{摘 \quad 要}
 \renewcommand{\bibname}{参考文献}
 
\titleformat{\chapter}[hang]{\huge}{\textbf{第~\thechapter~章}}{0.2cm}{\textbf}

\title{人机强交互环境对儿童青少年认知控制的影响与干预研究}

\date{\today}


\begin{document}
\maketitle

\tableofcontents

\newpage


每一章用不同的颜色
\begin{enumerate}
\item 引言 :  黑色
\item 人机强交互环境 : \textcolor{blue}{蓝色字体}
\item 认知控制: \textcolor{magenta}{ 玫红色字体}
\item 在线社交:  \textcolor{olive}{橄榄色字体}
\item 电子游戏: \textcolor{violet}{紫罗兰色字体}
\item 干预:\textcolor{cyan}{ 浅蓝色字体}
\end{enumerate}

\color{black}
\chapter{引言 (待补充)}
 本节中,我们将...



\color{black}
\chapter{人机强交互环境 (吴敏)
}
%\chapter{人机强交互环境}
\section{定义}
\section{特征}
\section{影响}


\color{black}
\chapter{认知控制  {(宋明、吴敏)}
}
%\chapter{认知控制}
\section{定义}
\section{国内外研究现状}
\section{内容}
\section{特征}

\color{black}
\chapter{网络成瘾 {}
}
%\color{navy}


\section{起源}



\section{定义}




\section{分类}
%\subsection{定义}
\subsection{在线游戏成瘾}
\subsection{社交成瘾}
\subsection{互联网赌博}
%\subsection{网络色情}




\color{black}
\chapter{在线社交对儿童青少年认知控制的影响}
%\chapter{在线社交对儿童青少年认知控制的影响}
\section{ 在线社交兴起的背景及对人们工作生活的影响 (可稍后补充)}
\section{在线社交的定义、特征特征}
\section{哪些方面与儿童青少年认知发生联系,结合到认知控制的影响是怎样的}
\section{新的值得关注的现象:社交成瘾?}


\color{black}
\chapter{电子游戏对儿童青少年认知控制的影响}
%\chapter{电子游戏对儿童青少年认知控制的影响}
\section{电子游戏定义、分类、人机强交互环境下的电子游戏有什么新特征}
\section{对认知控制的影响(双刃剑、好坏两方面)}
\section{最为关注的社会问题:游戏成瘾}
\subsection{游戏成瘾的定义、表现}
\subsection{游戏成瘾影响,侧重于认知控制方面的影响}

\color{black}
\chapter{干预(待补充 )}
\color{cyan}

(待补充)



\bibliographystyle{unsrt}

\bibliography{brain}
%\end{CJK}
\end{document}

